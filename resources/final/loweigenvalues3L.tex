% LaTeX class specification.  Any changes will not appear in the final version
\documentclass[12pt,letterpaper, reqno]{amsart}

% Only the following packages are available.
% If these are not sufficient, then you must submit a special request
% for a new package to be includade.  (We apologize for any inconvenience,
% but we cannot maintain these documents indefinitely without such restrictions).
\usepackage{amssymb,latexsym, amsmath, amsxtra, xy}
\usepackage[dvips]{graphics}

% The aimpl style file is necessary in order for the latex version
% to faithfully mimic the appearance of the problem list after it is
% uploaded. The file aimpl.sty should be in same directory as this file.

\usepackage{aimpl_2eq}

% New macros.
% If at all possible, please do not define any new macros.  Each
% macro you introduce has the potential to cause problems with the
% long-term maintenance of the problem list.

%%%%%%%%%%%%%%%%%%%%%%%%%%%%%%%%%%%%%%%%%%%%%%%%%%%%%%%%%%%%%%%%%%%%%
%
% Document begins below.
%
% Lines beginning with ZZZZ indicate standard material
%
%%%%%%%%%%%%%%%%%%%%%%%%%%%%%%%%%%%%%%%%%%%%%%%%%%%%%%%%%%%%%%%%%%%%%

\begin{document}
\title{Low Eigenvalues of Laplace and Schroedinger Operators}
\author{Edited by Rupert Frank and Richard Laugesen}

\urlstub{loweigenvalues}

\maketitle

Following are brief statements of some problems raised during the
AIM Workshop ``Low Eigenvalues of Laplace and Schroedinger
Operators,'' May 22--26, 2006, and the MFO Oberwolfach, February 9--13, 2009. The name of the participant who
mentioned the problem is stated in most cases, along with a brief
reference to more information. This participant is not necessarily
the original proposer of the problem in the literature, of course.

The problem statements given below include some editorial additions
which may not reflect the views of the person who
mentioned the problem.

\section{Polya and Related Inequalities}

Consider eigenvalues of the Dirichlet Laplacian on a bounded domain
$\Omega \subset \R^n$:
\[
-\Delta u_j = E_j u_j
\]
in $\Omega$, with $u_j = 0 $ on $\partial \Omega$.
Assume $n \geq 2$.


\begin{problemblock}

\begin{problem}[1.1] \by{Michael Loss, Timo Weidl}
The Polya Conjecture claims
that the Weyl asymptotic formula provides a lower bound:
\[
E_j \geq (2\pi)^2 (n/ |S^{n-1}| |\Omega|)^{2/n} j^{2/n} , \qquad j =
1,2,3,\ldots .
\]
\end{problem}
\begin{distinguishedremark}
The conjecture remains open even for $j=3$.
\end{distinguishedremark}


\begin{remark}
The best partial result known is with a factor of $n/(n+2)$ (which
is less than $1$) on the right hand side, as one deduces by
estimating $E_j \leq E_J$ in the following inequality due to Li and Yau,
\[
\sum_{j=1}^J E_j \geq \frac{n}{n+2} (2\pi)^2 (n/ |S^{n-1}|
|\Omega|)^{2/n} J^{(n+2)/n} , \qquad J=1,2,3,\ldots .
\]
\end{remark}

\begin{remark}
Berezin proved in 1972 that
\[
\sum_j (E - E_j)_+^\sigma \leq \frac{|\Omega|}{(2\pi)^n} \int_{\R^n} (E
- |p|^2)_+^\sigma \, dp , \qquad \sigma \geq 1, \quad E>0.
\]
The cases $0 \leq \sigma < 1$ remain open. The Polya conjecture is exactly the case $\sigma=0$.
The inequality for $\sigma=1$ implies the Li--Yau inequality via the Legendre transform.
\end{remark}

\end{problemblock}

\begin{problemblock}


\begin{problem}[1.2] \by{Timo Weidl}
Can one strengthen the Li--Yau result by
including a correction term, perhaps involving the surface area of
the boundary?
\end{problem}

\begin{distinguishedremark}
This has been done for the discrete
Laplacian on domains in a lattice; J. K. Freericks, E. H. Lieb, D. Ueltschi, Phase separation due to quantum mechanical correlations, Phys. Rev. Lett. 88, 106401 1-4 (2002).
\end{distinguishedremark}

\begin{remark}
There is a result by Melas of Li--Yau type with
corrections involving moments of inertia rather than surface
area, see A. Melas, A lower bound for sums of eigenvalues of the Laplacian,
Proc. Amer. Math. Soc. 131 (2003), 631--636.
\end{remark}

\begin{remark}
(February 2009) The result of Melas has been strengthened by inclusion
of a correction term involving the surface area of the boundary; see
H. Kovarik, S. Vugalter, T. Weidl, Two-dimensional Berezin--Li--Yau
inequalities with a correction term, Comm. Math. Phys. 287 (2009),
959--981. An earlier improvement for $\gamma\geq 3/2$, involving
a notion of effective boundary, is due to T. Weidl, Improved
Berezin--Li--Yau inequalities with a remainder term, in: Spectral
theory of differential operators, T. Suslina and D. Yafaev (eds.),
Amer. Math. Soc. Transl. Ser. 2, 225 (2008). For $\gamma<3/2$ further
improvements seem possible and desirable.
\end{remark}

\end{problemblock}


\begin{problemblock}

There are analogues of the Polya and Li--Yau inequalities under
Neumann boundary conditions, with the inequality signs reversed. The Polya Conjecture
remains open for Neumann boundary conditions for $j \geq 2$, except it was recently proved for $j=2$ in two dimensions by A. Girouard et al., J. Diff. Geometry, to appear. The analogue
of Li--Yau was proved by Pawel Kroeger; see P. Kroeger, Upper bounds for the Neumann eigenvalues on a bounded domain in Euclidean space, J. Funct. Anal.  106  (1992),  no. 2, 353--357, and also A. Laptev Dirichlet and Neumann
eigenvalue problems on domains in Euclidean spaces, J. Funct. Anal.
151 (1997), 531--545. 

\begin{problem}[1.3] \by{Timo Weidl} 
Can one strengthen the Kroeger result by including a correction term?

\end{problem}

\end{problemblock}

\begin{problemblock}
\begin{problem}[1.4]\by{Timo Weidl} 
 The questions raised above are meaningful in the presence of a magnetic field. For more information and some progress see Problem 2.55 below.
\end{problem}
\end{problemblock}

\begin{problemblock}
For $\sigma\geq 2$ the mapping
$$
   r_{\sigma}: E\mapsto E^{-\sigma-d/2}\bigg(\frac{|\Omega|}{(2\pi)^n}\int_{\mathbb{R}^n}(E-|p|^2)^{\sigma}dp -\sum_j(E-E_j)_{+}^{\sigma}\bigg)
$$
is non-increasing. This was proved in E. M. Harrell and L. Hermi, Differential inequalities for
Riesz means and Weyl-type bounds for eigenvalues, {J. Funct.
Analysis} {254} (2008), 3173--3191, using the trace identities of E. M. Harrell and J. Stubbe, On trace identities and universal eigenvalue estimates for some partial differential
operators, {Trans. Amer. Math. Soc.} {349} (1997),
1797--1809, and Universal bounds and
semiclassical estimates for eigenvalues of abstract Schroedinger
operators, preprint 2008, available as arXiv:0808.1133.

\begin{problem}[1.5] \by{Evans Harrell, Joachim Stubbe}
According to Weyl's asymptotic formula
$r_{\sigma}(E)$ tends to zero as $E$ tends to infinity and
therefore $r_{\sigma}(E)\geq 0$, which is the
Berezin--Li--Yau-inequality. Can one strengthen this bound in the
trace identity of Harrell--Stubbe to obtain correction
terms involving the surface area of the boundary?
\end{problem}

\begin{distinguishedremark}
 For the
Laplacian with periodic boundary conditions a similar monotonicity
property holds (for details see E. M. Harrell and J. Stubbe, Trace identities for
commutators, with applications to the distribution of
eigenvalues, preprint 2009, available as arXiv:0903.0563). In this case the
search for the correction term is related to the famous Gauss
circle problem (or lattice point problem).
\end{distinguishedremark}
\end{problemblock}


\begin{problemblock}

\begin{problem}[1.55] \by{Evans Harrell, Joachim Stubbe}
Prove monotonicity results like in Problem 1.5 for higher
order operators (e.g. clamped plate problem) and fractional powers
of Laplacians.
\end{problem}

\begin{distinguishedremark}
For some results on $\sqrt{-\Delta}$ see E. M.
Harrell and S. Yildirim Yolcu, Eigenvalue
inequalities for Klein-Gordon Operators, accepted for publication
in J. Funct. Analysis, leading to Berezin-Li-Yau inequalities for
these operators.
\end{distinguishedremark}

\end{problemblock}

\begin{problemblock}
For $p>0$ let
$$
    M_{p}(J):=\bigg(\frac{n+2p}{n}\frac1{J}\sum_{j=1}^JE_j^p\bigg)^{\frac1{p}}
$$
and for $p=0$ define
$$
    M_{0}(J):=e^{\frac{2}{n}}\bigg(\prod_{j=1}^JE_j\bigg)^{\frac1{J}}.
$$
According the Weyl asymptotic formula, for all $p\geq 0$,
$$
    M_{p}(J) \sim (2\pi)^2(n/|S^{n-1}||\Omega|)^{2/n}J^{2/n}
$$
as $J\rightarrow\infty$. In E. M. Harrell and J. Stubbe, On trace identities and
universal eigenvalue estimates for some partial differential
operators, {Trans. Amer. Math. Soc.} {349} (1997),
1797--1809, it has been shown that
$$
    M_1^2(J)-M_{2}(J)\geq \frac1{4}(E_{J+1}-E_J)^2 (\geq 0)
$$
and
$$
    M_1(J)-\sqrt{M_1^2(J)-M_{2}(J)}\leq E_J\leq E_{J+1}\leq
    M_1(J)+\sqrt{M_1^2(J)-M_{2}(J)}.
$$
Both inequalities are sharp in the Weyl limit. For extensions to
other $M_p(J)$ see E. M. Harrell and J. Stubbe, Universal bounds and
semiclassical estimates for eigenvalues of abstract Schroedinger
operators, preprint 2008, available as arXiv:0808.1133.

\begin{problem}[1.6]\by{Evans Harrell, Joachim Stubbe}
For $p>0$ find an upper bound of
the form
$$
    M_p^{2p}(J)-M_{2p}^p(J)\leq C(p,\Omega)E_1^{2p}J^{2p\kappa}
$$
with $\kappa<2/n$.
\end{problem}

\end{problemblock}

\begin{problemblock}


\begin{problem}[1.65]\by{Evans Harrell, Joachim Stubbe}
With the above notations does
$$
    E_J\leq M_p(J)
$$
hold for all $J$ and all $p\geq 0$?
Can one find $\Omega$ and $J$ such that the inequality
$$
    M_1^2(J)-M_{2}(J)\geq \frac1{4}(E_{J+1}-E_J)^2
$$
is saturated?
\end{problem}


\end{problemblock}


\section{Lieb--Thirring Inequalities}

Write $E_1 < E_2 \leq E_3 \leq \cdots \leq 0$
for the eigenvalues of $-\Delta - V$ on
$L^2(\R^n)$, meaning
\[
(-\Delta - V)u_j = E_j u_j .
\]
The eigenfunctions $u_j$ represent bound states with energies $E_j$.
For simplicity we assume $V \geq 0$. Assume $n \geq 1$.

The Lieb--Thirring inequality can be written as
\[
\sum_j |E_j|^\gamma \leq L_{n,\gamma} \int_{\R^n} V^{\gamma + n/2} \,
dx ,
\]
This inequality holds (with a constant $L_{n,\gamma}$ independent of $V$) iff the parameter $\gamma$ satisfies $\gamma\geq 1/2$ if $n=1$, $\gamma>0$ if $n=2$ and $\gamma\geq 0$ if $n\geq 3$. The case $\gamma = 0$ (counting eigenvalues)
is the Cwikel--Lieb--Rozenblum Inequality (CLR).

In other words
\[
{Tr} \left( -\Delta - V \right)_-^\gamma \leq
\frac{C_{n,\gamma}}{(2\pi)^n} \int_{\R^n} \int_{\R^n} (|p|^2 -
V(x))_-^\gamma \, dp dx ,
\]
where
\[
C_{n,\gamma} = \frac{L_{n,\gamma}}{L^{cl}_{n,\gamma}} \qquad
{ and } \quad L^{cl}_{n,\gamma} = \frac{1}{(2\pi)^n}
\int_{\R^n} (|p|^2-1)_-^\gamma \, dp .
\]
The constant $L^{cl}_{n,\gamma}$ is called the semiclassical
Lieb--Thirring constant.

Note that $C_{n,\gamma} \geq 1$ always, by the Weyl asymptotics, and
that $C_{n,\gamma}$ is decreasing in $\gamma$ for each fixed $n$, by
the Aizenman--Lieb monotonicity result.

To start with, let us summarize some known results on the constants
$C_{n,\gamma}$, along with conjectures about best (smallest) values
of $C_{n,\gamma}$.

\htmladdimage{Table summarizing results and conjectures about C(n,gamma)}{http://aimath.org/pl/loweigtable.jpg}

References to the results in the table and
to many of the questions below can be found in the lecture notes by
Michael Loss and Timo Weidl, and in the survey paper by Dirk
Hundertmark (which further states some better estimates on
$C_{n,\gamma}$ for special values of $n$ and $\gamma$).

For February 2009 updates see J. Dolbeault, A. Laptev, M. Loss, Lieb--Thirring inequalities with improved constants, J. Eur. Math. Soc. 10 (2008), and R. L. Frank, E. H. Lieb, R. Seiringer, Number of bound states of Schroedinger operators with matrix-valued potentials, Lett. Math. Phys. 82, 107 (2007).

Now we state open problems on Lieb--Thirring
inequalities.



\begin{problemblock}
\begin{problem}[2.1] \by{Richard Laugesen}
Must an optimal potential $V$ exist, for
those Lieb--Thirring inequalities in which the best constant is not
known? In particular this question is open for $n=1$ and
$\frac{1}{2} < \gamma < \frac{3}{2}$.
\end{problem}
\end{problemblock}

\begin{problemblock}
\begin{problem}[2.13]
A restricted version of the problem asks: within the class of
potentials having $m$ bound states (where $m \geq 1$ is given), does
an optimal potential exist?
\end{problem}
\end{problemblock}




\begin{problemblock}

\begin{problem}[2.16] \by{Richard Laugesen}
If an optimal potential exists, then does it have just a single bound state?
\end{problem}

\begin{distinguishedremark}
In other
words, does $-\Delta - V$ have just a single eigenvalue?
\end{distinguishedremark}


\begin{remark}
When $n=1$
and $\frac{1}{2} < \gamma < \frac{3}{2}$, the natural conjecture is
that the optimal potential is the one found by J. B. Keller when he
determined the best constant in $|E_1|^\gamma \leq C \int_\R
V^{\gamma + 1/2} \, dx$ (see J. Mathematical Phys. 2:262--266,
1961).
\end{remark}

\begin{remark}
This ``single bound state'' conjecture is due to Lieb and Thirring,
1976. In dimension $n=1$, the conjecture is known to be true in the
endpoint cases $\gamma=1/2$ (in which case $V$ is a delta function)
and $\gamma=3/2$ (in which case $V$ is a transparent or
reflectionless potential).
\end{remark}

\end{problemblock}

\begin{problemblock}

\begin{problem}[2.2] \by{Eric Carlen} 
Does there exist a bound of the
form $\sum_j |E_j|^\gamma \leq C |E_1|^\gamma$? 
\end{problem}

\begin{distinguishedremark}
 Here the factor $C$
could depend on $n,\gamma$, and on the integrability of a power of
$V$ sufficient to guarantee that the lefthand side is finite.
\end{distinguishedremark}

\end{problemblock}

\begin{problemblock}


\begin{problem}[2.23] \by{Rafael Benguria} 
The use of Korteweg--de Vries
(KdV) integrable system methods when $n=1, \gamma=3/2$, suggests
that one might similarly study Lieb--Thirring inequalities for the
linear equation associated with the Benjamin--Ono equation (another
integrable system).
\end{problem}

\begin{distinguishedremark}
Tomas Ekholm, Rupert Frank and Dirk Hundertmark
made progress during the Workshop already, by obtaining the analog
of the Aizenman--Lieb ``monotonicity toward best constants'' result.
The Lax pair for the Benjamin--Ono equation can be found for example
in R.L. Anderson and E. Tafflin, The Benjamin-Ono equation
-Recursivity of linearization maps- Lax pairs,
Letters in Mathematical Physics, 9 (1985), 299--311. See also,
D.J. Kaup and Y. Matsuno, The inverse scattering transform
for the Benjamin--Ono equation,
Studies in applied mathematics 101 (1998), 73--98.
\end{distinguishedremark}

\end{problemblock}

\begin{problemblock}
The best constant when $n=1,
\gamma=1$, is due to Eden--Foias
(see A. Eden and C. Foias, A simple proof of the
generalized Lieb--Thirring inequalities of one--space dimension,
Journal of mathematical analysis and applications, 162 (1991), 250--254.)
More precisely, they proved a Sobolev inequality, which then gives a
Lieb--Thirring inequality via the Legendre transform.

\begin{problem}[2.26] \by{Rupert Frank} 
Can one find a more direct proof of this Lieb--Thirring
inequality?
\end{problem}
\end{problemblock}

\begin{problemblock}
\begin{problem}[2.3]
Can one sharpen the Eden--Foias bound by including correction terms in their argument?
\end{problem}

\begin{remark}
(February 2009) An operator-valued version of the Eden--Foias bound
has been proved by J. Dolbeault, A. Laptev, M. Loss, Lieb-Thirring
inequalities with improved constants, J. Eur. Math. Soc. 10 (2008). By
the `lifting of dimension'-argument this result leads to the best
known values for the constants in the Lieb--Thirring inequalities for
$\gamma\geq 1$ if $n=1$ and for $\gamma\geq 1/2$ if $n\geq 2$.
\end{remark}

\end{problemblock}

\begin{problemblock}
\begin{problem}[2.33] \by{Timo Weidl}
Can one find a way to directly
estimate the sum of the eigenvalues, without going through the
Birman--Schwinger transformation (which counts the
eigenvalues rather than summing them)?
\end{problem}
\end{problemblock}

\begin{problemblock} \name{The Ovals Problem}
Consider a smooth closed curve $\gamma$ of length $2\pi$ in
${\mathbb R}^3$, and let  $\kappa(s)$ be its curvature as a function
of arclength. The curve determines the one-dimensional Schroedinger
operator $H_C=-d^2/ds^2 + \kappa^2$ acting on $2\pi$-periodic
functions. This operator appears in the equation for the tension of
a smooth, elastic, inextensible loop [5], and in connection with a
Lieb--Thirring inequality in one dimension [4]; similar
Schroedinger operators with quadratic curvature potentials have
been studied in connection with quantum mechanics on narrow channels
[2], Dirac operators on the sphere [3], and curvature-driven flows
describing the motion of interfaces in reaction-diffusion
equations [1].

\begin{conjecture}[2.4]
The principal eigenvalue $e(\gamma)$ is
minimal when $\gamma$ is a circle, where it takes the value $1$.
\end{conjecture}

\begin{distinguishedremark}
This question is open even for planar loops that enclose convex sets
(`ovals').
\end{distinguishedremark}

\begin{remark}
 It is known that the value $e(\gamma)=1$ is
attained for an entire family of planar curves whose curvature is
given by $\kappa(s) = \bigl(\alpha^2\cos^2 s + \alpha^{-2}\sin^2 s
\bigr)^{-1}$. When $\alpha\to 0$, these curves collapse onto two
straight line segments of length $\pi$ joined at the ends. The
inequality $e(\gamma)\ge 1$ has recently been shown for curves in
some neighborhood of the family [5], and for curves satisfying
additional geometric constraints [6]. The best universal lower bound
on $e(\gamma)$ that is currently known is $0.6085$ [6].

\end{remark}

\begin{remark}
Several participants at the Workshop had worked on this problem
previously (including Benguria, Loss, Burchard, Thomas, and Linde).
All agreed that classical Calculus of Variations techniques may be
exhausted at this point, and that rearrangement techniques seem to
fail. Linde and Burchard claimed that minimizers can be shown to
exist, and should be convex, but could conceivably contain one
corner, or two corners joined by a straight line segment.  Benguria
pointed to the family of putative minimizers (which look like
ellipses in polar coordinates) as evidence that the problem may have
a hidden affine symmetry. Carlen, Mazzeo, and Benguria proposed to
search for geometric flows that drive $e(\gamma)$ towards its
minimum. The affine curvature flow [7] was mentioned as a promising
candidate. Rapti and Lee proposed to analyze the Euler--Lagrange
equation using ODE methods. Laugesen suggested applying the
Birman--Schwinger transformation, after which the conjecture becomes
that the largest eigenvalue of the operator $T = \kappa (d^2/ds^2 +
\gamma)^{-1} \kappa$ is larger than $1$, for each constant
$0<\gamma<1$. Equivalently, take $\gamma=1$ and try to show the
largest eigenvalue of $T$ is larger than $1$, when $T$ acts on
functions $\psi$ with $\kappa \psi$ orthogonal to $\sin s$ and $\cos
s$. The hope is that a good choice of trial function (in the
variational principle for the largest eigenvalue) might suffice to
prove this conjecture.
\end{remark}

REFERENCES for the ovals problem


[1] E. M. Harrell and M. Loss. On the Laplace operator
penalized by mean curvature. Commun. Math. Phys. 195:643-650
(1998).

[2] P. Exner, E. M. Harrell and M. Loss. Optimal
eigenvalues  for some Laplacians and Schroedinger operators
depending on curvature. Oper. Theory Adv. Appl. 108:47-58 (1999).

[3] T. Friedrich. A geometric estimate for
a periodic Schroedinger operator. Colloq. Math. 83:209-216 (2000).

[4] R. D. Benguria and M. Loss. Connection between
the Lieb--Thirring conjecture for Schroedinger operators and an
isoperimetric problem for ovals on the plane. Contemporary Math.
362:53-61 (2004).

[5] A. Burchard and L. E. Thomas. On an isoperimetric inequality
for a Schroedinger operator depending on the curvature of a loop.
J. Geometric Analysis 15:543-563 (2005).

[6] H. Linde. A lower bound for the ground state energy of a
Schroedinger operator on a loop.,
Proc. Amer. Math. Soc. 134 (2006), 3629--3635.


[7]  B. Andrews. The affine curve-lengthening flow.
Crelle J. Reine Angew. Math. 506:43-83 (1999).

\end{problemblock}

\begin{problemblock}


\begin{problem}[2.5] \by{Timo Weidl}
For $n=2, \gamma=0$, can one
prove a Cwikel--Lieb--Rozenblum Inequality that involves a
logarithmic correction factor?
\end{problem}

\begin{distinguishedremark}
Without some such correction factor,
the inequality fails, since any nontrivial attractive potential has
at least one bound state.

(February 2009) This problem has been solved in H. Kovarik, S. Vugalter, T. Weidl, Spectral estimates for two-dimensional Schroedinger operators with application to quantum layers. Comm. Math. Phys. 275 (2007), no. 3, 827--838.
\end{distinguishedremark}

\end{problemblock}

\begin{problemblock}
\begin{problem}[2.53] \by{Timo Weidl}
 Can one obtain improved
Lieb--Thirring constants when working on a domain $\Omega$ rather
than on all of $\R^n$? For example, can one obtain a boundary
correction term?
\end{problem}
\end{problemblock}

\begin{problemblock}\name{Magnetic Schroedinger operators on a domain}
Consider the Dirichlet Laplacian in a domain
in $\R^n$. The technique of iteration-in-dimension gives sharp
Lieb--Thirring constants for arbitrary magnetic fields for $\gamma
\geq 3/2$ and any $n \geq 2$. (See the final part of A. Laptev and
T. Weidl, Sharp Lieb--Thirring inequalities in high
dimensions, Acta Mathematica 184 (2000), 87-111.) For $1/2 \leq
\gamma < 3/2$ one also gets estimates uniform in the magnetic field,
but the constant is (probably) not sharp. With the same approach,
the results of D. Hundertmark, A. Laptev and T. Weidl (New
bounds on the Lieb--Thirring constants, Inventiones Math. 140
(2000), 693-704) carry over to magnetic operators; see the remark at
the end of that paper.

The sharp Li--Yau bound (corresponding to $\gamma=1$) has been
proved by L. Erdos, M. Loss and V. Vougalter (Diamagnetic
behavior of sums of Dirichlet eigenvalues, Ann. Inst. Fourier
(Grenoble) 50 (2000), 891--907) for constant magnetic fields.

\begin{problem}[2.55] 
 Does
this bound hold true for arbitrary magnetic fields for $1\leq\gamma<3/2$?
\end{problem}
\end{problemblock}

\begin{problemblock}
\begin{problem}[2.58]
For $\gamma=0$, does the Polya conjecture hold true for tiling domains in the
presence of magnetic fields?
\end{problem}


\begin{remark}
(February 2009) The answer to the latter question is negative for
constant magnetic fields. Indeed, the sharp constant in the corresponding
lower bound for $0\leq\gamma<1$ was found in R. L. Frank, M. Loss,
T. Weidl, Polya's conjecture in the presence of a constant
magnetic field. J. Eur. Math. Soc., to appear.
\end{remark}

\end{problemblock}


\begin{problemblock} \name{Magnetic Schroedinger operators on $\R^n$}
Consider Lieb--Thirring bounds for magnetic
Schroedinger operators on all of $\R^n$. In all cases where the
sharp constant is known, either the magnetic field is not relevant
(dimension $n=1$) or the value of the constant is independent of the
magnetic field ($\gamma\geq 3/2$ and $n\geq 2$ as above, where the
sharp constant equals the classical constant).

\begin{problem}[2.6] \by{Timo Weidl} 
Can the magnetic field change the optimal value of the
Lieb--Thirring constant in the remaining cases?
\end{problem}

\begin{remark}
(February 2009) The magnetic field can change the optimal
value at most by an explicit factor depending only on $\gamma$ and
$d$; see R. L. Frank A simple proof of Hardy-Lieb-Thirring
inequalities. Comm. Math. Phys., to appear.
\end{remark}

\begin{remark}
This question is rather speculative, because we do not know the
sharp constants even in the non-magnetic case. But let us put
forward the following more specific version:
\end{remark}
\end{problemblock}

\begin{problemblock}
\begin{problem}[2.63]
Can one construct a counterexample to the Lieb--Thirring conjecture
that the optimal constant is the classical one for $n=3, \gamma=1$,
by using a suitable magnetic field?
\end{problem}

\end{problemblock}


\begin{problemblock} \name{Generalization to manifolds}
\begin{problem}[2.66] \by{Eric Carlen}
Do there exist Lieb--Thirring inequalities on manifolds?
As a basic first question, do the critical exponents
($\gamma=\frac{1}{2}$ when $n=1$, and $\gamma=0$ when $n=2$) depend
on the geometry?
\end{problem}

\begin{remark}
Some references to get started here are A. A. Ilyin,
Lieb--Thirring inequalities on the $N$-sphere and in the
plane, and some applications, Proc. London Math. Soc. (3) 67
(1993), 159--182; and Lieb--Thirring integral inequalities and
their applications to attractors of Navier-Stokes equations, Sb.\
Math. 196 (2005), 29--61. A classic reference for applications to
turbulence is E. Lieb, On characteristic exponents in
turbulence, Comm. Math. Phys. 92 (1984), 473--480.
\end{remark}

\begin{remark}
February 2009: Intuition from recent results on continuous trees
suggest that the critical exponents depend on both the local and global
dimension of the manifold (see T. Ekholm, R. L. Frank, H. Kovarik,
Eigenvalue estimates for Schroedinger operators on metric trees,
arXiv:0710.5500v1.)
\end{remark}

\begin{remark}
Analogues of Lieb-Thirring inequalities on tori and spheres have been proved in E. Harrell and J. Stubbe, Trace identities for commutators, with applications to the distribution of eigenvalues,    arXiv:0903.0563v1.
\end{remark}

\end{problemblock}


\begin{problemblock} \name{Reverse Lieb-Thirring Inequality} 
\begin{problem}[2.7]\by{Mark Ashbaugh}
Investigate reverse Lieb--Thirring inequalities.
\end{problem}

\begin{distinguishedremark}
For dimension $n=1$, Damanik and Remling have proved a
Reverse Lieb--Thirring Inequality in the subcritical range $0 <
\gamma \leq \frac{1}{2}$ (Schroedinger operators
with many bound states, Duke Math. J. 136 (2007), 51--80).  Sharp constants
seem not to be known. A Reverse
Cwikel--Lieb--Rozenblum Inequality for the eigenvalue counting function for dimension $n=2$ in the critical case $\gamma=0$ has been proved by A. Grigor'yan, Yu. Netrusov, S.-T. Yau, Eigenvalues of elliptic operators and geometric applications, Surveys in Differential Geometry IX (2004), 147-218.
\end{distinguishedremark}

\end{problemblock}


\begin{problemblock} \name{Powers of the Laplacian}
\begin{problem}[2.73] \by{Rupert Frank}
Can one prove a critical Lieb--Thirring inequality for arbitrary powers of
the Laplacian?
\end{problem}

\begin{distinguishedremark}
That is, one wants
\[
{tr\,} \left( (-\Delta)^s - V \right)_-^\gamma \leq
L_{\gamma,n} \int_{\R^n} V_+^{\gamma + n/2s} \, dx
\]
for $\gamma = 1 - n/2s > 0$. Such an inequality is known for $s$ a
positive integer by work of Netrusov--Weidl.
\end{distinguishedremark}


\begin{remark} \by{Timo Weidl}
Regardless of whether these operators have
physical significance, the higher order situation can help shed
light on what makes the second-order case work.
\end{remark}

\end{problemblock}


\begin{problemblock} \name{Hardy-Lieb-Thirring inequality}
\begin{problem}[2.76] \by{Rupert Frank}
Can one prove a Lieb--Thirring bound with a Hardy
weight, on the half-line?
\end{problem}

\begin{distinguishedremark}
 That is, one wants
\[
{tr\,} \left( -\frac{d^2}{dr^2} - \frac{1}{4r^2} - V
\right)_{\! -}^{\theta/2}  \leq C_\theta \int_0^\infty V(r)
r^{1-\theta}\, dr
\]
for $0 < \theta \leq 1$.
\end{distinguishedremark}
\begin{remark}
The inequality is known for $\theta=1$
(Lieb--Thirring). For $\theta=0$ it fails (although note that if it
were true, it would resemble Bargmann's inequality).
\end{remark}

\begin{remark}
(February 2009) The inequality for all $0 < \theta \leq 1$ has been
proved in T. Ekholm, R. L. Frank,  Lieb-Thirring inequalities on
the half-line with critical exponent. J. Eur. Math. Soc. 10 (2008),
no. 3, 739 - 755. The sharp constant
$C_\theta$ is not known, and there is not even a conjecture for it.
\end{remark}

\end{problemblock}


\begin{problemblock} \name{Cwikel-Lieb-Rozenblum bounds and heat kernel inequalities}
Let $Y$ be the Yamabe operator, or conformal Laplacian, on the
Euclidean ``round'' sphere $(S^n,g)$. That is, $Y = \Delta_{S^n} +
\frac{n}{2} \left( \frac{n}{2}-1 \right)$, where $\Delta_{S^n}$
denotes the Laplace--Beltrami operator on $S^n$.

Consider a positive smooth function $W$ on $S^n$, normalized so that
$\int_{S^n} W^{n/2}=$volume of the round sphere. Define
$Y_W=W^{-1/2}YW^{-1/2}$, acting on $L^2(S^n,g)$.

\begin{conjecture}[2.81] \by{Carlo Morpurgo}
 For $n \geq 3$,
$$
\max_{t>0} \left\{ t^{n/2} {Tr}[e^{-tY_W}] \right\} \leq
\max_{t>0} \left\{ t^{n/2} {Tr}[e^{-tY}]   \right\} .
$$
\end{conjecture}

\begin{distinguishedremark}
Note that the eigenvalues of $Y_W$ are the same as the eigenvalues
of $W^{-(n+2)/4}Y W^{(n-2)/4}$ acting on $L^2(S^n, W g)$, which is
the natural Yamabe operator in the metric $Wg$.

In other words we are looking for the best constant $C(W)$ in the
inequality
$$
{Tr}[e^{-tY_W}] \leq \frac{C(W)}{t^{n/2}},  \qquad t>0,
$$
and the conjecture states that this constant is attained precisely
by the right side of the equation in Conjecture 2.81.
\end{distinguishedremark}

\begin{remark}
If Conjecture 2.81 is true then we can considerably improve the known
CLR bounds, at least in low dimensions, noting that for a given
positive potential $V$, the eigenvalues of the Birman--Schwinger
operator $V^{-1/2}\Delta V^{-1/2}$ are the same as those of $Y_W$,
with $W=(V\circ\pi) |J_\pi|^{2/n}$, $\pi$ being the stereographic
projection and $J_\pi$ its Jacobian.
\end{remark}
\end{problemblock}

\begin{problemblock}

\begin{conjecture}[2.83] \by{Carlo Morpurgo}
 If $n \geq 4$ then the
function $f_W(t) = t^{n/2} {Tr}[e^{-tY_W}]$ is decreasing in
$t$.
\end{conjecture}


\begin{remark}
An asymptotic expansion $f_W(t) \sim a_0(W)+ t a_1(W)+\ldots$ holds
as $t \to 0$, with $a_0(W)= (4\pi)^{-n/2} \int_{S^n} W^{n/2}$ and
with $a_1(W)$ written explicitly in terms of the total curvature.
Hence Conjecture 2.83 would imply (equality in) Conjecture 2.81 for $n
\geq 4$, because Conjecture 2.81 normalizes the constant term $a_0(W)$
in the expansion.
\end{remark}

\begin{remark}
It is known that $a_1(W)$ is negative for $n \geq 5$, zero for
$n=4$, and positive for $n=3$, so that Conjecture 2.83 fails for small
$t$ when $n=3$.

On the other hand, Conjecture 2.83 holds for large $t$ and any $n \geq
3$, since the known sharp lower bound $\lambda_0(W) \geq
\lambda_0(1)= \frac{n}{2} \left( \frac{n}{2}-1 \right)$ for the
lowest eigenvalue of $Y_W$ implies that $f_W(t)$ is decreasing when
$t > \left( \frac{n}{2}-1 \right)^{-1}$.
\end{remark}

\begin{remark}
Conjecture 2.83 is true if $W \equiv 1, n \geq 4$.
\end{remark}


\end{problemblock}

\section{Gap Inequalities}

 Consider eigenvalues of the
Dirichlet Laplacian on a bounded convex domain $\Omega \subset \R^n$
with convex potential $V$:
\[
(-\Delta + V) u_j = \lambda_j u_j 
\]
in $\Omega$, with $u_j = 0 $ on $\partial \Omega$.
Assume $n \geq 1$. Notice the operator is written with $+V$, not
$-V$ like in the previous section.

\begin{problemblock}\name{The Gap Conjecture}

\begin{conjecture}[3.1]\by{Van den Berg}
\[
\lambda_2 - \lambda_1 \geq \frac{3\pi^2}{d^2} , \qquad d={diam}(\Omega) ,
\]
with equality holding when $n=1, V \equiv 0$.
\end{conjecture}

\begin{distinguishedremark}
In dimensions $n \geq
2$, the inequality should be strict, with equality holding only in
the limit as the domain degenerates to an interval.
\end{distinguishedremark}

\begin{remark}
In dimension $n=1$ the conjecture has been completely proved by
Richard Lavine (1994).
\end{remark}

\begin{remark}
In dimensions $n \geq 2$, the best partial result says that
$\lambda_2 - \lambda_1 \geq \pi^2/d^2$, which is missing the desired
factor of $3$ on the righthand side. The first proof of this result
used $P$-function techniques based on the maximum principle. The
second proof adapted the methods of Weinberger, who resolved the
analogous Neumann gap problem long ago.
\end{remark}

\begin{remark}
For an extended treatment of the problem and many references, see
Mark Ashbaugh's introduction The Fundamental Gap on the AIM
website. Also see the overhead transparencies of Rodrigo Banuelos's
talk.
\end{remark}
\end{problemblock}


\begin{problemblock}

\begin{problem}[3.15] \by{Richard Lavine} 
Can one expand the class of
potentials for which the gap inequality holds, in one dimension? 
\end{problem}

\begin{distinguishedremark}
It
is known for convex potentials, but also for single well potentials
with a centered transition point. See the write-up by Mark Ashbaugh
\end{distinguishedremark}

\end{problemblock}

\begin{problemblock}
\begin{problem}[3.2] \by{Richard Lavine}
Normalize the eigenfunctions
$u_j$ in $L^2$ and define $\langle V \rangle_j = \int_\Omega V u_j^2
\, dx$. Are these means  $\langle V \rangle_j$ an increasing
sequence as $j$ increases?
\end{problem}
\begin{distinguishedremark}
 The question is already interesting in
one dimension.
\end{distinguishedremark}

\end{problemblock}



\begin{problemblock}
\begin{problem}[3.25] \by{Richard Lavine}
Can one strengthen the gap
inequality by adding to its righthand side a term that involves $V$?
The question is already interesting in one dimension.
\end{problem}
\end{problemblock}

\begin{problemblock}
\begin{problem}[3.3] \by{Rodrigo Banuelos}
Can Lavine's approach be
extended to higher dimensions?
\end{problem}
\end{problemblock}

\begin{problemblock}

In dimensions $n \geq 2$, one
should try to understand whether genuine barriers exist to pushing
the $P$-function techniques beyond the known $\pi^2/d^2$ bound. One
seems to need to improve the log-concavity bound on the groundstate
$u_1$ (due to Brascamp--Lieb). That is, instead of just discarding
the Hessian of $\log u_1$ when it arises, on the grounds that it is
$\leq 0$, one seems to want to bound the Hessian strictly away from
$0$.

\begin{problem}[3.35] \by{Mark Ashbaugh}
 Can this be achieved by the methods of Brascamp--Lieb, or of
Korevaar?
\end{problem}
\end{problemblock}

\begin{problemblock}
\begin{problem}[3.4] \by{Antoine Henrot} 
The Gap Conjecture is already
very interesting in the case of vanishing potential $V \equiv 0$. A
possible approach is as follows.

(1) Prove the gap infimum $\inf_{\Omega \in \mathcal{O}} (\lambda_2-\lambda_1)$
is not attained, when $\mathcal{O}$ is the class of convex domains
with diameter $1$.

(2) Prove that minimizing sequences shrink to a segment of
length $1$.

(3) Prove that the gap for a sequence of shrinking domains
behaves like the gap of a one-dimensional Schroedinger
operator with convex potential (semiclassical limit arguments).

(4) Complete the proof using the results in the one dimensional case (Lavine's Theorem).

\end{problem}
\begin{distinguishedremark}
It seems that the last three points are OK. It remains to prove the first one!
\end{distinguishedremark}
\end{problemblock}




\begin{problemblock} \name{Operator-valued potentials}
\begin{problem}[3.5] \by{Helmut Linde} 
In order to prove the gap conjecture one could consider
the Laplacian on a two-dimensional domain as being a one-dimensional
operator with a matrix-valued potential. This makes it possible to
approach the problem via a sequence of simplified ``toy models''.
For example, one can try to prove the gap conjecture first for very
special classes of matrix-valued potentials, like potentials that
have constant eigenvectors and whose eigenvalues are convex
functions. Then one could gradually generalize this theorem to
approach the ``real'' gap conjecture.
\end{problem}

\end{problemblock}

\begin{problemblock}
For magnetic Schroedinger operators,
the Gap Conjecture cannot hold as stated because the eigenvalue gap
can be reduced to zero by the introduction of a magnetic field.


\begin{problem}[3.55] \by{Timo Weidl and Richard Laugesen}
Can one still obtain a valid gap inequality by subtracting from the
righthand side a term depending on the magnetic potential $A$?
\end{problem}
\end{problemblock}

\begin{problemblock}
\begin{problem}[3.65] \by{Rodrigo Banuelos}  
Is the groundstate of $\sqrt{-\Delta}$ log-concave?
\end{problem}
\begin{distinguishedremark}
See
also the comments above on log-concavity of the groundstate of
$-\Delta$.
\end{distinguishedremark}
\end{problemblock}

\begin{problemblock} \name{The Hot Spots conjecture}
\begin{conjecture}[3.7] \by{Bernhard Kawohl} 
The first nontrivial eigenfunction of the Neumann Laplacian attains
its maximum and mimimum values on the
boundary of the convex domain $\Omega$.
\end{conjecture}

\begin{distinguishedremark}
 This has been proved only for some special classes of
domains. The analogous conjecture for the Dirichlet Laplacian would
be that the ratio $u_2/u_1$ attains its maximum and mimimum values
on the boundary of $\Omega$. Note $u_2/u_1$ satisfies Neumann
boundary conditions (by explicit calculation, assuming the boundary
is smooth) and satisfies a certain elliptic equation.
\end{distinguishedremark}
\end{problemblock}

\begin{problemblock}
Turn now from the Dirichlet boundary condition to the Robin
condition $\partial u / \partial \nu = -\alpha u$ (for some given
constant $\alpha > 0$, with $\nu$ denoting the outward normal).

\begin{problem}[3.75]\by{Robert Smits}
Is
the gap $\lambda_2 - \lambda_1$ minimal when $V=0$ and $\Omega$
degenerates to a segment having the same diameter as $\Omega$?

In one dimension, is the gap minimal when $V=0$ and $\Omega$ is a
segment? Can Lavine's methods be adapted to Robin boundary
conditions, in one dimension?
\end{problem}
\begin{distinguishedremark}
If one could prove the groundstate $u_1$ is log-concave, then
existing methods could be adapted to imply $\lambda_2 - \lambda_1
\geq \pi^2/d^2$, like is already known for the Neumann and Dirichlet
situations. Incidentally, the Rayleigh quotient for the gap can be
shown (like in the Dirichlet case) to equal
\[
\lambda_2 - \lambda_1 = \min_{\int_\Omega f u_1^2 \, dx = 0}
\frac{\int_\Omega |\nabla f|^2 u_1^2 \, dx}{\int_\Omega f^2 u_1^2 \,
dx} ,
\]
with the potential entering implicitly through the dependence of
$u_1$ on $V$.
\end{distinguishedremark}
\end{problemblock}


\end{document}
