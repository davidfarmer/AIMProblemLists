
% LaTeX class specification.  Any changes will not appear in the final version
\documentclass[12pt,letterpaper, reqno]{amsart}

% Only the following packages are available.
% If these are not sufficient, then you must submit a special request
% for a new package to be includade.  (We apologize for any inconvenience,
% but we cannot maintain these documents indefinitely without such restrictions).
\usepackage{amssymb,latexsym, amsmath, amsxtra, xy}
\usepackage[dvips]{graphics}

% The aimpl style file is necessary in order for the latex version
% to faithfully mimic the appearance of the problem list after it is
% uploaded. The file aimpl.sty should be in same directory as this file.

\usepackage{aimpl_2eq}

% The problem list should be arranged like a research paper, with
% introductory material followed be one or more main sections.
% Do not divide the sections into subsections.
% Each section could have its own introductory material.
% The main part of each section is a "problem block" which has the format
% [[see elsewhere for more documentation]]

% New macros.
% If at all possible, please do not define any new macros.  Each
% macro you introduce has the potential to cause problems with the
% long-term maintenance of the problem list.

\begin{document}
\title{The Riemann Hypothesis and related problems}
\author{Edited by J. Brian Conrey and David W Farmer}

\plversion{1.0}

\urlstub{rhrelated}

\maketitle

This document collects problems related to the Riemann Hypothesis (RH).
The focus is on
analogues of RH for other global L-functions, weaker statements
about the zeros of zeta- and L-functions, and problems motivated
by equivalences to RH.

The Riemann Hypothesis concerns the nontrivial zeros
of an L-function $L(s)$.
By a nontrivial zero we mean a zero of the completed L-function
$\Lambda_L(s)$, equivalently, a zero of $L(s)$ which is not at the
location of a pole of one of the $\Gamma$-factors that appears in
the functional equation for $L(s)$.

\section{The zeta function and other global L-functions}


The Riemann Hypothesis for an L-function $L(s)$ is the assertion that the
nontrivial zeros of $L(s)$ lie on the
critical line.  

For historical reasons there are
names given to the Riemann hypothesis for various sets
of $L$-functions.  For example, 
the Generalized Riemann Hypothesis (GRH) is
the Riemann Hypothesis for all Dirichlet $L$-functions.
More examples collected below.

In certain applications there is a fundamental distinction
between nontrivial zeros on the real axis and 
nontrivial zeros with a positive imaginary part.
Here we use the adjective  ``modified'' to indicate
a Riemann Hypothesis except for the possibility
of nontrivial zeros on the real axis.  Thus, the
Modified Generalized Riemann Hypothesis (MGRH) is
the assertion that all nontrivial zeros of 
Dirichlet $L$-functions lie either on the 
critical line or on the real axis.
A real zero which is very close to the point $s=1$ is
called a Landau-Siegel zero.  

Precise statements of various Riemann Hypotheses are given below.

\begin{problemblock}\name{The Riemann Hypothesis}
\begin{problem}[1.1] The nontrivial zeros of the Riemann Zeta function,
$\zeta(s)$ lie on the critical line $\sigma=\frac12$.
\end{problem}

\begin{distinguishedremark}
The Riemann Hypothesis is sometimes phrased in terms of the
zeros in the critical strip, or the
complex (meaning, non-real) zeros.  For the Riemann zeta function
these are equivalent statements, but there exist L-functions
whose nontrivial zeros do not all lie on the real line. 
\end{distinguishedremark}

\end{problemblock}

\begin{problemblock}\name{The Generalized Riemann Hypothesis(GRH)}
\begin{problem}[1.2]
Riemann Hypothesis is true, and in addition the
nontrivial zeros of
all Dirichlet $L$-functions
lie on the critical line $\sigma=\frac12$.
\end{problem}

\begin{remark}GRH is occasionally called ``Piltz conjecture,'' but the
conjecture of a Riemann Hypothesis for Dirichlet L-functions is
usually viewed as obvious generalization which should not
be attributed to a particular person.
\end{remark}

\begin{remark}GRH is the conjecture of a Riemann Hypothesis for
all degree 1 L-functions.
\end{remark}
\end{problemblock}

\begin{problemblock}\name{The Modified Generalized Riemann Hypothesis}
\begin{problem}[1.3]
Riemann Hypothesis is true, and in addition the
nontrivial zeros of
all Dirichlet $L$-functions
lie either on the real line or on the critical line $\sigma=\frac12$.
\end{problem}
\end{problemblock}


\begin{problemblock}\name{The Extended Riemann Hypothesis}
\begin{problem}[1.4]
The nontrivial zeros of the Dedekind zeta function
of any algebraic number field lie on the critical line.
\end{problem}


\begin{distinguishedremark}
Note that ERH includes RH because the Riemann zeta function is
the Dedekind zeta function of the rationals.
\end{distinguishedremark}

\end{problemblock}

\begin{problemblock}\name{The Grand Riemann Hypothesis}
\begin{problem}[1.5]
Nontrivial zeros of all $L$-functions
lie on the critical line.
\end{problem}

\begin{distinguishedremark}
The Grand Riemann Hypothesis is often phrased in terms of
all automorphic L-functions.  
\end{distinguishedremark}
\end{problemblock}

\begin{problemblock}\name{The Modified Grand Riemann Hypothesis}
\begin{problem}[1.6]
The nontrivial zeros of all $L$-functions
lie either on the real line or on the critical line.
\end{problem}

\end{problemblock}



\section{Weaker statements about zeros of L-functions}
The Riemann Hypothesis is the strongest possible statement
about the horizontal distribution of the nontrivial zeros
of an $L$-function.  In this section we collect various
weaker assertions. 

Throughout this section let $L(s)$ be an L-function
for which one might expect a Riemann Hypothesis to hold.

\begin{problemblock}\name{Quasi Riemann Hypothesis} 

\begin{problem}[2.1]
 $L(s)$
has no zeros in a half-plane
$\sigma > \sigma_0$, for some $\sigma_0 <1 $.
\end{problem}

\end{problemblock}


\begin{problemblock}\name{The Density Hypothesis}
Recall the standard notation:
$$
N(\sigma,T) = \#\{\rho=\beta+i\gamma\ :\ \beta \ge \sigma,\ 0<\gamma<T\}.
$$
The Riemann Hypothesis is equivalent to $N(\sigma,T)=0$ for $\sigma>\frac12$.

The Density Hypothesis is the assertion:
\begin{problem}[2.2] For all $\epsilon>0$,
$$
N(\sigma,T) = O(T^{2(1-\sigma)+\varepsilon}) .
$$

\end{problem}

\begin{distinguishedremark}
This is nontrivial 
only when $\sigma > \frac12$.
\end{distinguishedremark}

\begin{remark}
The importance of the Density Hypothesis is that, in terms of 
bounding the gaps between consecutive primes, the density hypothesis
appears to be as strong as the Riemann Hypothesis.
\end{remark}

\begin{remark}
Results on $N(\sigma,T)$ are generally obtained from mean
values of the zeta-function.  Further progress in this
direction, particularly for $\sigma$ close to $\frac12$, appears
to be hampered  by the great difficulty in estimating the
moments of the zeta-function on the critical line.

See Titchmarsh [88c:11049], Chapter 9, for an extensive discussion.

\end{remark}

\begin{remark}
The Density Hypothesis follows from the Lindel\"of Hypothesis.
\end{remark}
\end{problemblock}


\begin{problemblock}\name{The 100\% Hypothesis}
Let $N(T)$ denote the counting function of the nontrivial
zeros of $L(s)$, so $N(T)\sim \frac{d}{2\pi}T\log T$
where $d$ is the degree of $L(s)$.  And let
$N_0(T)$ denote the counting function of the
zeros of $L(s)$ on the critical line.
\begin{problem}[2.8]
The 100\% Hypothesis for $L(s)$
asserts that $N_0(T)\sim N(T)$ as $T\to \infty$.
\end{problem}

\begin{distinguishedremark}
An equivalent assertion is
$$
N(T)- N_0(T) = o(T\log T) ,
$$
which makes it clear that the 100\% Hypothesis still
allows quite a few zeros off the critical line.
\end{distinguishedremark}

\begin{remark}
The term ``100\% Hypothesis'' is not standard.
\end{remark}

\begin{remark}
In contrast to most of the other conjectures in
this section, the 100\% Hypothesis is not motivated
by applications to the prime numbers.   
Indeed, at
present there are no known consequences of
this hypothesis.
\end{remark}

\end{problemblock}

\section{Problems motivated by equivalences to the Riemann Hypothesis}

The following problems could offer insight into a possible approach
to the solving the Riemann Hypothesis.  Most of these problems
are motivated by items on the 
list of equivalences to the Riemann Hypothesis:  www.aimpl.org/pl/rhequivalences

\begin{problemblock}
\begin{problem}[3.1]
What is the constant in 
\htmladdnormallink{Baez-Duarte's equivalent}{http://aimpl.org/pl/rhequivalences/7.3} of RH?
\end{problem}
\end{problemblock}

\begin{problemblock}\name{Redheffer's matrix}
\begin{problem}[3.2]\by{Brian Conrey}
Does the Riemann Hypothesis imply that the ``nontrivial'' eigenvalues
of \htmladdnormallink{Redheffer's matrix}{http://aimpl.org/pl/rhequivalences/2.3}
are inside the unit circle?
\end{problem}
\end{problemblock}



\end{document}
