\section{Central Questions}
\label{sec:central}

There are three main families of combinatorial structures counted by
the Catalan combinatorics, which arise independently in three different
subjects.

\begin{enumerate}
\item Let $W$ be a finite Coxeter group, and let $T$ be the generating
set of \emph{all} reflections ($T$ is defined as the set of conjugates of
a standard Coxeter generating set $S$). Let $\ell$ denote the word length
on $W$ with respect to $T$. This function induces a partial order on $W$
by setting $a\leq b$ whenever $\ell(b)=\ell(a)+\ell(a^{-1}b)$. The Hasse
diagram of this poset is just the Cayley graph of $(W,T)$, directed away
from the identity element $1$.

Let $c$ be a Coxeter element of $W$. The interval $[1,c]$ in
this poset is called the poset of noncrossing partitions
$NC_W$. (This is well-defined, since Coxeter elements form a conjugacy
class.) $NC_W$ in its full generality was defined independently
by David Bessis \cite{bessis:dual} and Tom Brady and Colum Watt
\cite{brady,brady-watt:kpi1} in order to study the geometric group theory
of braid groups. However, important special cases of the generalization
had been considered earlier by Vic Reiner \cite{reiner} and Philippe Biane
\cite{biane}. In the type $A$ case, Roland Speicher showed that this poset
lies at the heart of the subject of free probability in operator algebras
\cite{speicher:survey}. The study of the type $A$ case is classical and
goes back to Kreweras \cite{kreweras}. The survey paper \cite{simion}
by Rodica Simion gives a comprehensive view of the classical noncrossing
partitions, and the survey \cite{mccammond:noncrossing} by Jon McCammond
gives a more modern overview of the subject.

\item For every finite Coxeter group $W$, Sergey Fomin and Andrei
Zelevinsky have defined a simplicial complex $\Delta_W$ called the
\emph{simplicial associahedron} of type $W$. Let $\Phi$ be the root
system corresponding to $W$, and let $\Phi^+$ and $\Pi$ be a choice
of positive roots and simple roots, respectively. Then $\Delta_W$ is
defined as a flag complex on the set of  almost positive roots
$\Phi_{\geq -1}:= \Phi^+ \cup (-\Pi)$. Their original construction
\cite{fomin-zelevinsky:ysystems} applied only to crystallographic
root systems, but the definition may be uniformly generalized to all
root systems (see the notes \cite{fomin-reading:survey}). The complex
$\Delta_W$ generalizes the well-known associahedron (or Stasheff polytope)
in type $A$, and the well-known cyclohedron (or Bott-Taubes polytope)
in type $B$. For more information, see Section \ref{sec:cluster}.

Related to this are the Cambrian lattices of Nathan Reading
\cite{reading:cambrian}. To each orientation of the Coxeter diagram of
$W$, he associates a lattice which is a quotient of the weak order on
$W$. Conjecturally, each of these Cambrian lattices is an orientation
of the $1$-skeleton of the simple associahedron, the dual complex
to $\Delta_W$. Cambrian lattices may be regarded as a generalization of
the classical Tamari lattices, and this idea has also been considered
by Hugh Thomas in type $B$ \cite{thomas}.


\item In the case that $W$ is a Weyl group (that is, a crystallographic
Coxeter group), Postnikov has suggested how to define a poset of
\emph{nonnesting partitions} $NN_W$ (see remarks in \cite{reiner}). Given
a crystallographic root system $\Phi$ with positive roots $\Phi^+$,
the root order $(\Phi^+,\leq)$ is a partial order on $\Phi^+$,
where $\alpha\leq\beta$ if and only if $\beta-\alpha$ is in the positive
integer span of $\Pi$.

To each antichain $\A$ (set of pairwise-incomparable elements) in
the root order, associate the vector subspace $\cap_{\alpha\in\A}
\alpha^{\perp}$, which is the intersection of the orthogonal hyperplanes
to the corresponding roots. Then $NN_W$ is defined as the poset of
antichains under reverse inclusion of subspaces.

The antichains may also be interpreted as order ideals (or order
filters) in the root order, and Cellini and Papi have shown that these
are in bijection with nilpotent ideals of a Borel subalgebra of the
corresponding semisimple Lie algebra \cite{cellini-papi}. One may define
a \emph{different} partial order on the antichains via inclusion of ideals,
and this poset describes the structure of the chambers within the dominant
cone of the Shi hyperplane arrangement \cite{shi}.

\end{enumerate}

 \begin{problemblock}

\begin{problem}[1.1]
\label{central:one}
Explain the numerology. The cardinality of $NC_W$, the cardinality of
$NN_W$ and the number of facets of $\Delta_W$ are all equal to the Catalan
number $\Cat(W)$. The rank numbers of $NC_W$, the height numbers of $NN_W$
(in general, $NN_W$ is not graded), and the $h$-vector of $\Delta_W$ are
all the same, given by the Narayana numbers (for which there is no
known closed formula, in general).
\end{problem}

 The enumerative coincidences are quite
extensive, and quite mysterious, as there is  still no theoreretical
connection between these objects.  In fact, only for $NN_W$ and its
relatives is there \emph{any} proof whatsoever of the enumerative formulas
that is not case-by-case, using the finite type classification.
\end{problemblock}

\begin{problemblock}
\begin{problem}[1.2]
Find bijections between these objects which preserve the numerology. Is
there some theoretical algebraic framework behind the scenes, as
yet undiscovered? David Bessis has suggested a notion of ``dual''
Coxeter systems \cite{bessis:dual}. Is there a way to formalize this
notion? The exponents of $W$ are one below the corresponding degrees of
the fundamental polynomial invariants of $W$ (see \cite{humphries}). Does
the number $\Cat(W)$ have any significance in an invariant theory context?
\end{problem}

\begin{remark}
 There are two remarkable enumerative refinements of the Catalan combinatorics, each in a different direction.
\begin{enumerate}
\item Fr\'ed\'eric Chapoton has defined a two variable generating function
on each of the three main families (the $M$-triangle on noncrossing
partitions, the $F$-triangle on the associahedron, and the $H$-triangle
on nonnesting partitions), and conjectured precise algebraic relationships
between these functions \cite{chapoton:one,chapoton:two}. This gives very
refined enumerative correspondences between these objects, and is strong
evidence for the existence of hidden structural relationships. Explain
Chapoton's formulas.

\item Christos Athanasiadis and Vic Reiner have described an enumerative
correspondence between $NC_W$ and $NN_W$ that refines the Narayana numbers
\cite{athanasiadis-reiner}. Both of these posets may be injected into the
lattice of hyperplane intersections $\Pi_W$ of the corresponding Coxeter
arrangement. For $\pi$ in $NC_W$ let $f(\pi)$ be the fixed subspace
of $\pi$, and for $\A$ in $NN_W$, let $g(\A)$ be the intersection of
hyperplanes $\cap_{\alpha\in\A} \alpha^{\perp}$, as before. The result
states that the filters of $f$ and $g$ over any $W$-orbit in $\Pi_W$
are equinumerous.

The proof is case-by-case, using computer in the exceptional types. Find
a theoretical proof. Is there a natural statistic on $\Delta_W$ that
agrees with this refinement of the Narayana numbers? Is there a way to
express this statistic within the context of Chapoton's $M$-triangle,
$F$-triangle and $H$-triangle generating functions?
\end{enumerate}
\end{remark}

\begin{remark}The recent work of Nathan Reading on Coxeter-sortable elements
\cite{reading} gives an explicit bijection between $NC_W$ and the
facets of $\Delta_W$, however the proof of this bijection is currently
case-by-case (see Problem \ref{prob:c-sorted}). Also, Tom Brady and Colum
Watt have given a new definition of $\Delta_W$ in terms of noncrossing
partitions \cite{brady-watt}. This may provide some connection between
the structure of $NC_W$ and $\Delta_W$.
\end{remark}

\end{problemblock}


\begin{problemblock}

\begin{problem}[1.3]\label{central:two} What are the largest natural domains of definition for the families $NC_W$, $NN_W$ and $\Delta_W$, and for their corresponding applications?\end{problem}

\begin{remark}
In a sense, the broadest setting possible for the numerology is  finite
groups generated by pseudoreflections. (A pseudoreflection is a unitary
operator on an $n$-dimensional complex vector space whose eigenvalues
are $0$ with multiplicity $n-1$, and $-1$ with multiplicity $1$.) It is
a classical result of Shephard and Todd that the ring of invariants of
a group $W$ is a polynomial ring precisely when the group is of this
type. And in this case the sequence of degrees $d_1,d_2,\ldots,d_n$
of fundamental invariants is unique \cite{shephard-todd}.

In the general (complex) case, David Bessis suggests that the Catalan number should be
\begin{equation*}
\Cat(W):= \prod_{i=1}^n  \frac{h+d_i}{d_i},
\end{equation*}
where we set $h$ equal to the highest degree $d_n$. This agrees with our earlier definition in the real types. However, this may apply only when $W$ is a duality group (or a well-generated group), since otherwise $\Cat(W)$ may fail to be an integer. See the paper \cite{bessis:complex} by David Bessis for more information.
\end{remark}

\begin{remark}
The noncrossing partitions are currently the most general of the Catalan families. The poset $NC_W$ is defined for all finite Coxeter groups, and the definition makes sense in principle for any finitely generated Coxeter group (although the definition may not be unique when $W$ is infinite \cite{mccammond-etal}). David Bessis and Ruth Corran gave a combinatorial realization of $NC_W$ for an infinite class of complex reflection groups in \cite{bessis-corran}, and Bessis has suggested a uniform definition for $NC_W$ whenever $W$ is a well-generated complex reflection group \cite{bessis:complex}.
\end{remark}

\begin{remark}
Can one generalize free probability beyond types $A$ and $B$? The combinatorics of free probability is naturally expressed in terms of the type $A$ noncrossing partitions \cite{speicher:survey}, and some work has been done on a type $B$ free probability \cite{biane-goodman-nica}. Does it make sense to generalize further? One would presumably need to express Roland Speicher's work on multiplicative functions \cite{speicher} in the completely general case. See Problem \ref{prob:freeprobability}.
\end{remark}

\begin{remark}
Explain the theory of cluster algebras in infinite types. See Problem \ref{prob:infinitetypes}.
\end{remark}

\begin{remark}
 The most glaring case of this problem is the seeming dependence of $NN_W$ and its relatives on the crystallographic structure of $W$. When $W$ is a Weyl group, there are amazing enumerative correspondences with the other Catalan objects (see the remarks following Problem \ref{central:one}). But there is currently no idea how to generalize these objects to the noncrystallographic types.

To what extent can the nonnesting partitions and root order be generalized to noncrystallographic types? Presumably, there are objects which can not be generalized in their current form, such as Lie algebras and affine hyperplane arrangements. Generalize where possible, and explain where there are essential barriers to this generalization. Cathy Kriloff and Arun Ram have dealt with some of these issues in studying the representation theory of noncrystallographic types \cite{kriloff-ram}.

Fr\'ed\'eric Chapoton's conjecture gives a way to define the $H$-triangle for all finite Coxeter groups \cite{chapoton:two}. What object is it counting in the noncrystallographic types?
\end{remark}

\end{problemblock}

\begin{problemblock}

\begin{problem}[1.4]
\label{central:three}
What are the most natural generalizations of the families $NC_W$, $NN_W$, and $\Delta_W$? Classical combinatorics is full of enumerative generalizations of the Catalan numbers. Which of these is relevant in the reflection group setting?
\end{problem}

Define the \emph{Fuss-Catalan numbers}
\begin{equation*}
\Cat^{(k)}(W):= \prod_{i=1}^n \frac{kh+e_i+1}{e_i+1},
\end{equation*}
where $k$ is a positive integer. In type $A$, these generalize the classical Fuss numbers and the Catalan numbers \cite{fomin-reading,hilton-pederson}. As seen from the formula, $\Cat^{(k)}(W)$ is a very natural generalization of the Catalan numbers in the reflection group context. Recently these numbers  have shown up in all three of the Catalan families.

\begin{remark}
Drew Armstrong has defined a generalization of the noncrossing partitions $NC_W^{(k)}$, called the \emph{$k$-divisible noncrossing partitions} \cite{armstrong}. This is a graded join-semilattice which is counted by $\Cat^{(k)}(W)$. Call the rank numbers the \emph{Fuss-Narayana numbers}. In types $A$ and $B$, $NC_W^{(k)}$ is isomorphic to the poset of $k$-divisible noncrossing set partitions (partitions in which each block has size divisible by $k$).
\end{remark}

\begin{remark}
Sergey Fomin and Nathan Reading have defined a simplicial complex $\Delta^{(k)}_W$ which is a generalization of the simplicial associahedron \cite{fomin-reading}. The facets of $\Delta^{(k)}_W$ are counted by the Fuss-Catalan numbers, and the entries of the $h$-vector are given by the Fuss-Narayana numbers. In types $A$ and $B$, this complex is defined in terms of $(k+2)$-angulations of a regular polygon, and has been studied independently by Eleni Tzanaki \cite{tzanaki}.
\end{remark}

\begin{remark}
The Fuss-Catalan numbers appear in many places in the $NN_W$ family of objects. Let $W$ be a finite Weyl group. Christos Athanasiadis suggested the definition of the Fuss-Narayana numbers in this context, and proved that these numbers count several objects, including positive regions in a certain affine deformation of the Coxeter hyperplane arrangement, as well as co-filtered multichains of ideals in the root order \cite{athanasiadis:cat,athanasiadis:nar}. Mark Haiman has shown that the Fuss-Catalan numbers count orbits in the quotient $\check{Q}/(kh+1)\check{Q}$ of the coroot lattice $\check{Q}$ \cite{haiman:conjectures}, and Eric Sommers has encountered these numbers in the study of Lie algebras \cite{sommers}.
\end{remark}
\end{problemblock}

\begin{problemblock} 	\begin{problem}[1.5]
Repeat Problems \ref{central:one} and \ref{central:two} in this more general setting. Any theoretical relationships found between $NC_W$, $NN_W$, and $\Delta_W$, must generalize to explain the Fuss-Catalan combinatorics. Given that $\Cat^{(k)}(W)$ is naturally defined in terms of the exponents of $W$, is there an underlying algebraic framework that explains these numbers?
\end{problem}

\begin{remark}
Extend Fr\'ed\'eric Chapoton's $M$-triangle, $F$-triangle, and $H$-triangle to the Fuss-Catalan case. (Eleni Tzanaki has worked on this for the $H$-triangle.)
\end{remark}

\begin{remark}
What is the significance of these Fuss-Catalan objects in applications, for instance in Garside Structures, cluster algebras, or free probability? For example, the $k$-divisible noncrossing partitions may have some application to Problem \ref{prob:kdivisible}, in free probability.
\end{remark}

\begin{remark}
Is there a natural generalization of the poset of nonnesting partitions $NN^{(k)}_W$? In type $A$, one may take $k$-divisible nonnesting set partitions under refinement (mimicking $NC_{A_{n-1}}^{(k)}$). In the general case, perhaps this is isomorphic to a partial order on co-filtered multichains of ideals in the root order.
\end{remark}

\begin{remark}
Christos Athanasiadis and Stavros Garoufallidis have suggested a $q$-version of the Catalan combinatorics. See Problem \ref{prob:qcat} below.
\end{remark}

\end{problemblock}
