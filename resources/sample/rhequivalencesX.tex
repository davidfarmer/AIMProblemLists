
% LaTeX class specification.  Any changes will not appear in the final version
\documentclass[12pt,letterpaper, reqno]{aimpl}

% The problem list should be arranged like a research paper, with
% introductory material followed be one or more main sections.
% Do not divide the sections into subsections.
% Each section could have its own introductory material.
% The main part of each section is a "problem block" which has the format
% [[see elsewhere for more documentation]]

% New macros.
% If at all possible, please do not define any new macros.  Each
% macro you introduce has the potential to cause problems with the
% long-term maintenance of the problem list.

\newcommand{\htmladdnormallink}[2]{{#1}\footnote{#2}}

\begin{document}
\title{Equivalences to the Riemann Hypothesis}
\author{Edited by J. Brian Conrey and David W Farmer}

\urlstub{rhequivalences}

\maketitle

The Riemann Hypothesis(RH)
is the assertion that all of the nontrivial zeros of the Riemann
zeta function have real part equal to~$\frac12$.

The Riemann Hypothesis has been shown to be equivalent to an
astounding variety of statements in several different areas
of mathematics.  Some of those equivalences are nearly trivial.
For example, RH is equivalent to the nonvanishing of $\zeta(s)$ in the
half-plane $\sigma>\frac12$.  Other equivalences appear surprising
and deep.  Examples of both kinds are collected below.

For background on the Riemann zeta function and the Riemann Hypothesis,
see ??? and also the \htmladdnormallink{Riemann Hypothesis and related problems}{http://aimpl.org/pl/rhrelated/} list.

% Articles with equivalences that may not have appeared here yet:
%[96g:11111] [98f:11113] [96a:11085] [95c:11105]
%[94i:58155] [89j:15029] [87b:11084]

\section{Equivalences involving primes}

The main point of Riemann's original paper
is that the two sequences, of prime numbers on the one hand, and of zeros
of $\zeta$ on the other hand, are in duality. A precise mathematical
formulation of this fact is given by the so-called explicit formulas of
prime number theory (Riemann, von Mangoldt, Guinand, Weil). Therefore,
any statement about one of these two sequences must have a translation
in terms of the other.

Some statements, such as the random matrix conjecture for the
normalized neighbor spacing of the zeros,
or the existence of infinitely many twin primes, do not seem
to have a simple translation into a statement about the other
sequence.
But, as Riemann conjectured but did not prove, RH has a
simple formulation in terms of the prime numbers.

Let $Li$ be the ``Logarithmic integral'' function, defined by
$$
Li(x) := \int_0^x \frac{dt}{\log t},
$$
% check on the correct lower limit
the integral being evaluated in principal value in the neighborhood of $x=1$.
And let
\begin{align}
\pi(x) =\mathstrut & \sum_{p\le x} 1 \\
	=\mathstrut & \text{the number of primes } p\le x .
\end{align}
Here and throughout this document, $p$ always stands for a prime number.

\begin{problemblock}
\begin{rhequivalence}[1.1] The Riemann hypothesis is equivalent to
$$
\pi(x) = Li(x) + O(x^{1/2 + \epsilon})
$$
for any $\epsilon > 0$.
\end{rhequivalence}

Roughly speaking, this means that the first half of the digits of the
$n$-th prime are those of $Li^{-1}(n)$.

\end{problemblock}

\begin{problemblock}
von Koch, Acta Mathematica 24 (1901), 159-182, showed:
\begin{rhequivalence}[1.15] The Riemann hypothesis is equivalent to
$$
\pi(x) = Li(x) + O( \sqrt{x} \log x).
$$
\end{rhequivalence}

\remark{In 1976 L.Schoenfeld [56 15581b] gave a numerically explicit version of this equivalent form:
$$
|\pi (x) -li (x)| \leq \frac{\sqrt{x}\log x}{8\pi} \text{ for } \geq 2657.
$$
}
\end{problemblock}

\section{Averages of arithmetic functions}
These equivalent statements have the following shape:
$$
\sum_{n \leq x} f(n) = F(x) + O(x^{\alpha + \epsilon}), \quad x \rightarrow +\infty,
$$
where $f$ is an arithmetic function, $F(x)$ a smooth approximation to $\sum_{n \leq x} f(n)$, and $\alpha$ a real number.

\begin{problemblock}\name{The von Mangoldt function}

The von Mangoldt function $\Lambda (n)$ is defined
as $\log p$ if $n$ is a power of a prime $p$, and $0$ in the other cases. Define:
$$
\psi(x) := \sum_{n \leq x} \Lambda (n).
$$

\begin{rhequivalence}[2.1]
RH is equivalent to
$$
\psi (x) =x + O(x^{1/2 + \epsilon}),
$$
for every $\epsilon>0$.
\end{rhequivalence}
\end{problemblock}

\begin{rhequiv}[2.13]\label{equiv:psilog2}
RH is equivalent to
$$
\psi (x) =x + O(x^{1/2}\log^2 )
$$
\end{rhequiv}

\begin{problemblock} L.Schoenfeld [56 15581b] refined
Equivalence~\ref{equiv:psilog2} to the numerically explicit form
\begin{rhequivalence}[2.17]
RH is equivalent to
$$
|\psi (x) -x| \leq \frac{x^{1/2}\log^2x}{8\pi} \text{ for } > 73.2.
$$
\end{rhequivalence}
\end{problemblock}

\begin{problemblock}\name{The M\"obius function}
The M\"obius function $\mu (n)$ is defined as $(-1)^r$ if
$n$ is a product of $r$ distinct primes, and as $0$ if the
square of a prime divides $n$. Define:
$$
M(x) := \sum_{n \leq x} \mu (n).
$$
Littlewood proved the following two equivalences.

\begin{rhequivalence}[2.2]\label{equiv:mobius_sum}
RH is equivalent to
$$
M (x) \ll x^{1/2 + \epsilon},
$$
for every positive $\epsilon$
\end{rhequivalence}
\end{problemblock}

\begin{rhequiv}[2.25]
RH is equivalent to
$$
M (x) \ll x^{1/2} \exp (A \log x/\log\log x),
$$
for some $A>0$.
\end{rhequiv}

\begin{problemblock}\name{Redheffer's matrix}
The Redheffer matrix $A(n)$ is the $n\times n$  matrix of 0's and 1's
defined by $A(i,j) = 1$ if $j = 1$ or if $i$ divides $j$, and
$A(i,j) = 0$ otherwise.

Redheffer proved that $A(n)$ has $ n-[n \log 2]-1$ eigenvalues equal to 1.
Also,     A has a real eigenvalue  (the spectral radius) which is approximately $\sqrt{n}$,
     a negative eigenvalue which is approximately $-\sqrt{n}$ and
     the remaining eigenvalues are small.

The connection with the Riemann Hypothesis is that
$$
\det A(n) = \sum_{1\le j\le n} \mu(j) .
$$
Therefore by Equivalence~\ref{equiv:mobius_sum},

\begin{rhequivalence}[2.3] The Riemann Hypothesis is equivalent to
$\det(A) = O(n^{1/2+\epsilon})$ for every $\epsilon > 0$.
\end{rhequivalence}

\begin{remark}

Barrett  Forcade, Rodney, and Pollington
[MR 89j:15029]  give an easy proof of Redheffer's theorem.
They also prove that the spectral radius of $A(n)$ is
$=n^{1/2}+\tfrac 12\log n+O(1).$  See also the paper of
Roesleren [ MR 87i:11111].
\end{remark}


\begin{remark}

Vaughan [MR 94b:11086] and [MR 96m:11073]
determines the dominant eigenvalues with an error term $O(n^{-2/3})$ and shows that the nontrivial eigenvalues are  $\ll(\log n)^{2/5}$ (unconditionally), and $\ll\log\log(2+n)$ on the Riemann Hypothesis.
\end{remark}

\begin{remark}\by{Brian Conrey}
It is possible that all the nontrivial eivenvalues lie in the unit disc.
\end{remark}
\end{problemblock}


\section{Large values of arithmetic functions}

RH is equivalent to several inequalities of the following type:
$$
f(n) < F(n),
$$
where $f$ is an ``arithmetic'' or ``irregular'' function,
and $F$ an ``analytic'' or ``regular'' function.

\begin{problemblock}\name{The sum of divisors of $n$}

Let
$$
\sigma(n) = \sum_{d|n} d
$$
denote the sum of the divisors of $n$.

Also let
$$
 H_n = \sum_{j=1}^n \frac{1}{j}
$$
denote the $n$th harmonic number, and
\begin{align}\label{eqn:eulergamma}
\gamma =\mathstrut& \lim_{n\to\infty} H_n - \log n  \\
\approx \mathstrut& 0.577215...
\end{align}
denote Euler's constant

G.~Robin [86f:11069] showed that

\begin{rhequivalence}[3.1]The Riemann Hypothesis is
equivalent to
$$
\sigma(n) < e^\gamma n \log\log n
$$
for all $n\ge 5041$.
\end{rhequivalence}

That inequality does not leave much to spare, for Gronwall showed
$$
\limsup_{n\to\infty} \frac{\sigma(n)}{n \log\log n} = e^\gamma ,
$$
and Robin showed unconditionally that
$$
\sigma(n) < e^\gamma n \log\log n + 0.6482 \frac{n}{\log\log n},
$$
for $n\ge 3$.
\end{problemblock}

\begin{problemblock}J. Lagarias [arXiv:math.NT/0008177] elaborated on Robin's
work and showed that
\begin{rhequivalence}[3.15]The Riemann Hypothesis is
equivalent to
$$
\sigma(n) < H_n + \exp(H_n)\log(H_n)
$$
for all $n\ge 2$.
\end{rhequivalence}

\remark{By the definition~\eqref{eqn:eulergamma} of $\gamma$,
Lagarias' and Robin's inequalities are the same to leading order.}

\end{problemblock}

\begin{problemblock}\name{The Euler totient function}

The Euler function $\phi (n)$ is defined as the number of positive integers not exceeding $n$ and coprime with $n$.
That is, it is the multiplicative function which has
$\phi(p^n)=p^n-p^{n-1}$.

Also, let $N_k$ be the product of the first $k$ prime numbers.

\begin{rhequivalence}[3.2]The Riemann Hypothesis is
equivalent to
$$
\frac{N_k}{\phi(N_k)} > e^{\gamma} \log \log N_k,
$$
for all $k$.
\end{rhequivalence}
\end{problemblock}

\begin{rhequiv}[3.25]The Riemann Hypothesis is
equivalent to
$$
\frac{N_k}{\phi(N_k)} > e^{\gamma} \log \log N_k,
$$
for all but finitely many $k$.
\end{rhequiv}


\begin{problemblock}\name{The maximal order of an element in the symmetric group}

Let $g(n)$ be the maximal order of a permutation of $n$ objects,
$\omega(k)$ be the number of distinct prime divisors of the integer $k$
and $Li$ be the integral logarithm.

Massias, Nicolas and Robin [89i:11108] showed that
\begin{rhequivalence}[3.3]The Riemann Hypothesis is
equivalent to
$$
\log g(n) < \sqrt{Li^{-1}(n)} \quad \text{ for $n$ large enough}.
$$
\end{rhequivalence}
\end{problemblock}
\begin{rhequiv}[3.35]The Riemann Hypothesis is
equivalent to
$$
\omega(g(n)) < Li (\sqrt{Li^{-1}(n)}) \quad \text{ for $n$ large enough}.
$$
\end{rhequiv}

\section{Farey series}
 Let $r_v$ be the elements of the Farey sequence of order $N$, $v=1,2,\dots \Phi(N)$ where $\Phi(N)=\sum_{n=1}^N\phi(n)$.  Let $\delta_v=r_v-v/\Phi(N)$.

A good (put possibly out-of-date) bibliography on Farey sequences and RH
is available at {\tt http://people.math.jussieu.fr/$\sim$miw/telecom/biblio-Amoroso.html}.

\begin{rhequiv}[4.1]The Riemann Hypothesis is
equivalent to
$$\sum_{v=1}^{\Phi(N)} \delta_v^2\ll N^{-1+\epsilon}.$$
\end{rhequiv}
\begin{rhequiv}[4.15]The Riemann Hypothesis is
equivalent to
$$\sum_{v=1}^{\Phi(N)} |\delta_v|\ll N^{1/2+\epsilon}.$$
\end{rhequiv}

\begin{problemblock}\name{Amoroso's criterion}
Amoroso [MR 98f:11113] has proven the following interesting equivalent
to the Riemann Hypothesis.  Let $\Phi_n(z)$ be the $n$th cyclotomic
polynomial and let $F_N(z) = \prod_{n \le N}\Phi_n(z)$. Let
$$\tilde h(F_N) = (2\pi)^{-1}\int_{-\pi}^\pi \log^+|F(e^{i\theta})|~d\theta.$$
Then,
\begin{rhequivalence}[4.2]
$\tilde h(F_n) \ll N^{\lambda + \epsilon}$ is equivalent to the assertion that the Riemann zeta function does not vanish for ${\rm Re}{z} \ge \lambda + \epsilon$.
\end{rhequivalence}
\end{problemblock}

% put something here about Mikolas function

\section{Weil's positivity criterion}
Andr\'{e} Weil [MR 14,727e] proved the following explicit formula (see also A. P. Guinand [MR 10,104g] which
specifically illustrates the dependence between primes and zeros.
 Let $h$ be an
even function which is holomorphic in the strip $|\Im t|\le
1/2+\delta$ and satisfying $h(t)=O((1+|t|)^{-2-\delta})$ for some
$\delta>0$, and let
$$g(u)=\frac{1}{2\pi}\int_{-\infty}^\infty h(r)e^{-i u r}~dr.$$
Then we have the following duality between primes and zeros:
$$
\sum_{\gamma}h(\gamma)=2h(\tfrac{i}2)  -g(0) \log \pi
+\frac{1}{2\pi} \int_{-\infty}^\infty
h(r)\frac{\Gamma'}{\Gamma}(\tfrac14+\tfrac 12 i
r)~dr-2\sum_{n=1}^\infty \frac{\Lambda(n)}{\sqrt{n}}g(\log n).$$
 In this formula, a zero is written as $\rho=1/2+i\gamma$ where
$\gamma\in \mathbb C$; of course RH is the assertion that all of
the $\gamma$ are real. Using this duality Weil gave a criterion for RH.

\begin{problemblock}\name{Bombieri's refinement }
Bombieri [1 841 692]has given the following version of Weil's criterion.

Let
$$
\hat{g}(s)=\int_0^\infty g(x)x^{s-1}~dx.
$$

\begin{rhequivalence}[5.1]The Riemann Hypothesis is
equivalent to
$$\sum_{\rho}\hat{g}(\rho) \hat{\overline{g}}(1-\rho)>0$$
for every complex-valued $g(x)\in C_0^\infty(0,\infty)$
which is not identically 0.
\end{rhequivalence}
\end{problemblock}

\begin{problemblock}\name{Li's criterion}
 Xian-Jin Li [98d:11101] proved the following assertion:
\begin{rhequivalence}[5.2]The Riemann Hypothesis is
equivalent to
$\lambda_n\ge 0$ for all $n$, where
$$
\lambda_n =\sum_\rho (1-(1-1/\rho)^n),
$$
where the sum is over the zeros of the zeta function.
\end{rhequivalence}

Another expression for $\lambda_n$ is given by
$$
\lambda_n=\frac{1}{(n-1)!}\frac{d^n}{ds^n}(s^{n-1}\log \xi(s))\vert_{s=1}
$$
where
$$
\xi(s)=\frac12s(s-1)\Gamma(s/2)\zeta(s)
$$
\end{problemblock}

\begin{problemblock}
\begin{rhequivalence}[5.3]
The Riemann Hypothesis is equivalent to the equality
$$ \sum_{\rho} \frac{1}{|\rho|^2} = 2+\gamma-\log 4\pi
$$
where the sum is over all complex zeros $\rho=\beta+i\gamma$ of $\zeta(s)$ in the critical strip $0<\beta<1$.
\end{rhequivalence}
\end{problemblock}


\section{Function-theoretic properties of $\zeta$ }

\begin{problemblock}\name{Speiser's criterion }
A. Speiser (Math Annahlen 110 (1934) 514-521) prove
\begin{rhequivalence}[6.1]The Riemann Hypothesis is
equivalent to
the nonvanishing of the derivative $\zeta'(s)$ in the
left-half of the critical strip $0<\sigma< 1/2$.
\end{rhequivalence}

\begin{remark} Levinson and Montgomery [MR 54 5135]
gave an alternative, quantitative version of this result.
This led to Levinson's
[MR 58 27837] discovery of his method for counting zeros on the critical
line,  which he used to prove that at least 1/3 of the zeros of
$\zeta(s)$ are on the critical line.
\end{remark}

\end{problemblock}

\begin{problemblock}

\begin{rhequivalence}[6.2]The Riemann Hypothesis is
equivalent to
$$\Re \frac{\xi'(s)}{\xi(s)}>0$$
for $\Re s>1/2$.
\end{rhequivalence}

%  Need a reference and a refernece to recent work of Lagarias
\end{problemblock}

\begin{problemblock}

V. V. Volchkov [MR 96g:11111] proved
\begin{rhequivalence}[6.3]The Riemann Hypothesis is
equivalent to
$$
\int_0^\infty \int_{1/2}^\infty
 \frac{1-12y^2}{(1+4 y^2)^3}\log(|\zeta(x+iy)|)\, dx \, dy
=\pi\frac{3-\gamma}{32}
$$
\end{rhequivalence}
\end{problemblock}


\section{Function spaces }

Beginning with Wiener's paper, ``Tauberian Theorems'' in the Annals of
Math (1932) a number of functional analytic equivalences of RH have been
proven. These involve the completeness of various spaces.  M. Balazard has
written a survey on these developments (See Surveys
in Number Theory, Papers from the Millennial Conference on Number Theory,
A. K. Peters, 2003.)

\begin{problemblock}\name{The Beurling-Nyman Criterion}
Let $\mathcal N_{(0,1)} $ be the space of functions
$$
f(t)=\sum_{k=1}^n c_k \rho(\theta_k/t)
$$
for which $\theta_k\in (0,1)$ and such that $\sum_{k=1}^n c_k=0$.

\begin{rhequivalence}[7.1]The Riemann Hypothesis is
equivalent to the assertion that $\mathcal N_{(0,1)} $
is dense in  $L^2(0,1)$.
\end{rhequivalence}

\end{problemblock}

\begin{problemblock} Beurling [MR 17,15a] proved that the following
equivalences of a quasi-Riemann Hypothesis.
\begin{rhequivalence}[7.2] Suppose $1<p<\infty$.  The following
are equivalent:
\begin{enumerate}
\item{} $\zeta(s)$ has no zeros in $\sigma>1/p$,
\item{}   $\mathcal N_{(0,1)} $ is dense in  $L^p(0,1)$,
\item{} The characteristic function $\chi_{(0,1)}$ is in the closure
of $\mathcal N_{(0,1)} $   in  $L^p(0,1)$.
\end{enumerate}
\end{rhequivalence}
\end{problemblock}

\begin{problemblock}\name{Mollifiers}
Baez-Duarte [arXiv:math.NT/0202141]
proved
\begin{rhequivalence}[7.3] The Riemann Hypothesis is equivalent to
$$
\inf_{A_N(s)}\int_{-\infty}^\infty |1-A_N(1/2+it)\zeta(1/2+it|^2)~\frac {dt}{\frac 14 +t^2}
\to 0
$$
as $N\to \infty$,  where
the infimum is over all
Dirichlet polynomials of length $N$:
$$
A_N(s)=\sum_{n=1}^N \frac {a_n}{n^s}.
$$
\end{rhequivalence}
\end{problemblock}

\begin{problemblock}\name{Salem's criterion}
Let
$$
k_\sigma(x) =\frac{x^{\sigma-1}}{e^x+1}.
$$
R. Salem [MR 14,727a] proved an equivalence for the nonvanishing
of the zeta function on a vertical line:
\begin{rhequivalence}[7.4] The non-vanishing of $\zeta(s)$ on the
$\sigma$ --line is equivalent to the completeness in
$L^1(0,\infty)$ of $\{k_\sigma(\lambda x),\ \lambda>0\}$.
\end{rhequivalence}
\end{problemblock}

\section{The zeta function at the positive integers}

These equivalences involve $\zeta(n)$ at integers $n\ge 2$,
or the Bernoulli numbers via the identity
$$
\zeta\left(2k\right)=\frac{\left(-1\right)^{k+1}\left(2\pi\right)^{2k}}{2%
\left(2k\right)!}B_{2k}
$$



\begin{problemblock}\name{Riesz series}
M. Riesz (Sur l'hypoth\`{e}se de Riemann, Acta Math. 40 (1916), 185-190)
proved
\begin{rhequivalence}[8.1] The Riemann Hypothesis is equivalent to
$$
\sum_{k=1}^\infty \frac{(-1)^{k+1}x^k}{(k-1)!\zeta(2k)}\ll x^{1/2+\epsilon}
$$
for any $\epsilon>0$.
\end{rhequivalence}
\end{problemblock}

\begin{problemblock}\name{Hardy-Littlewood series}
Hardy and Littlewood (Acta Mathematica 41 (1918), 119 - 196) showed

\begin{rhequivalence}[8.2] The Riemann Hypothesis is equivalent to
 $$\sum_{k=1}^\infty \frac{(-x)^k}{k!\zeta(2k+1)}\ll x^{-1/4}$$
as $x\to \infty$.
\end{rhequivalence}
\end{problemblock}

 \begin{problemblock}\name{Carey's series}

\begin{rhequivalence}[8.3]  The Riemann Hypothesis is equivalent to the convergence of the series
$$
\sum_{n=0}^\infty\left(n+\frac{1}{2}\right)\left|\sum_{k=0}^n\frac{c_{2n+1,
2k+1}}{2k+2}\log\left(\frac{2k+1}{2k+2}\frac{\left(-1\right)^kB_{2k+2}\left(2
\pi\right)^{2k+2}}{2\left(2k+2\right)!}\right)\right|^2<\infty
$$
where $c_{m,r}$ denotes the coefficient of $x^r$ in the Legendre
polynomial of degree~$m$.
Specifically,
$$
c_{2n+1,2k+1}=\frac{\left(-1\right)^{n-k}\left(2n+2k+2\right)!}
{2^{2n+1} \left(n-k\right)!\left(n+k+1\right)!\left(2k+1\right)!}.
$$
\end{rhequivalence}

\end{problemblock}

\section{Analytic estimates}

\begin{problemblock}\name{Polya's integral criterion}
Polya (see Collected Works, Volume 2, Paper 102, section 7) gave a number of
integral criteria for Fourier transforms to have only real zeros.

Let
$$\Phi(u)=2\sum_{n=1}^\infty (2 n^4\pi^2e^{\frac 9 2 u}-3 n^2 \pi e^{\frac 5 2 u})e^{-n^2 \pi e^{2 u}}$$
so that
$$\xi(\tfrac12+i z)=\int_{-\infty}^\infty \Phi(t)e^{iz} \,dt.
$$
One of Polya's criteria gives

\begin{rhequivalence}[9.1] The Riemann Hypothesis is equivalent to
$$
\int_{-\infty}^\infty\int_{-\infty}^\infty
\Phi(\alpha)\Phi(\beta) e^{i(\alpha+\beta)x }
e^{(\alpha-\beta)y}(\alpha-\beta)^2~d\alpha ~d\beta \ge 0.
$$
\end{rhequivalence}
\end{problemblock}

\begin{problemblock}\name{Newman's criterion}
Charles Newman [MR 55 \#7944],
building on work of deBruijn [MR 12,250]
defined
$$
\Xi_\lambda(z)=\int_{-\infty}^\infty \Phi(t)e^{-\lambda t^2}e^{iz} ~dt.
$$
Note that $\Xi_0(z)=\Xi(z):=\xi(\frac12 + i z)$.

Newman proved that there exists a constant $\Lambda$ (with $-1/8 \le
\Lambda < \infty$) such that $\Xi_\lambda(z)$ has only real zeros if
and only if $\lambda\ge \Lambda$.

\begin{rhequivalence}[9.2] The Riemann Hypothesis is equivalent to
$\Lambda\le 0$.
\end{rhequivalence}

The constant $\Lambda$ (which Newman conjectured is equal to 0) is now called the deBruijn-Newman constant.

\begin{remark}
A. Odlyzko [MR 2002a:30046] has recently proven that
$-2.7·10^{-9}<\Lambda$.
\end{remark}

\begin{remark}
Conjectures for the distribution of gaps between zeros,
based on random matrix theory, imply that $\Lambda \ge 0$.
\end{remark}

\end{problemblock}

\begin{problemblock}\name{Grommer inequalities}
Let
$$-\frac{\Xi'}{\Xi}(t)=s_1+s_2t+s_3t^2+\dots.$$
Let $M_n$ be the matrix whose $i,j$ entry is $s_{i+j}$.
J. Grommer (J. Reine Angew. Math.  144 (1914), 114--165) proved

\begin{rhequivalence}[9.3] The Riemann Hypothesis is equivalent to
$\det M_n>0 $ for all $n\ge 1$.
\end{rhequivalence}


\begin{remark}
See also the paper of R. Alter [MR 36 1399].
\end{remark}
\end{problemblock}


\end{document}
