% LaTeX class specification.  Any changes will not appear in the final version
\documentclass[12pt,letterpaper, reqno]{amsart}

% Only the following packages are available.
% If these are not sufficient, then you must submit a special request
% for a new package to be includade.  (We apologize for any inconvenience,
% but we cannot maintain these documents indefinitely without such restrictions).
\usepackage{amssymb,latexsym, amsmath, amsxtra, xy}
\usepackage[dvips]{graphics}

% The aimpl style file is necessary in order for the latex version
% to faithfully mimic the appearance of the problem list after it is
% uploaded. The file aimpl.sty should be in same directory as this file.

\usepackage{aimpl_2}


% The problem list should be constructed like a research paper, with
% introductory material followed be one or more main sections.
% Do not divide the sections into subsections.
% Each section could have its own introductory material.
% The main part of each section is a "problem block" which has the format
% [[see elsewhere for more documentation]]

% New macros.
% If at all possible, please do not define any new macros.  Each
% macro you introduce has the potential to cause problems with the
% long-term maintenance of the problem list.
\newcommand{\Cat}{{\rm Cat}}
\newcommand{\A}{\mathcal A}
\newcommand{\freestar}{ \framebox[7pt]{$\star$} }

%%%%%%%%%%%%%%%%%%%%%%%%%%%%%%%%%%%%%%%%%%%%%%%%%%%%%%%%%%%%%%%%%%%%%
%
% Document begins below.
%
% Lines beginning with ZZZZ indicate standard material
%
%%%%%%%%%%%%%%%%%%%%%%%%%%%%%%%%%%%%%%%%%%%%%%%%%%%%%%%%%%%%%%%%%%%%%

\begin{document}
\title{Free Analysis}
%\author{ZZZZ Author goes here}

\maketitle

These problems arose during the AIM workshop ``Free Analysis'',
 June 19 - 23, 2006, organized by Dimitri Shlyakhtenko and Dan Voiculescu.

\section{ $X$-constants and free Poincare inequality }\by{Voiculescu}

\begin{problemblock}

 In a von Neumann algebra $M$
with a faithful normal trace-state $\tau$ let $X=X^{*}\in M$ and
let $1\in B\subset M$ be an infinite-dimensional von Neumann subalgebra
so that $B$ and $X$ are free in the algebraic sense and $M=W^{*}(X,B)$. 

Assume that $\partial_{X:B}$ is closable in $L^{2}(M,\tau)$ (this
is the case for instantce if $X$ is a free semicircular perturbation
$X=X_{0}+\varepsilon S$, with $S$ a semicircular free from $X_{0}$
and $B$). 

\begin{problem}
Under what conditions are the $L^{2}$ solutions of \[
\overline{\partial_{X:B}}u=0\]
in $L^{2}(B,\tau)?$ 
\end{problem}
\end{problemblock}

\begin{problemblock}

A related question about a stronger condition:

\begin{problem}
When does the free
Poincare inequality\[
C\Vert\partial_{X:B}\xi\Vert_{2}\geq\Vert\xi-E_{B}\xi\Vert_{2}\]
hold for $\xi\in B\langle X\rangle$?
\end{problem}

\end{problemblock}







\section{Large Deviations}
\by{Guionnet, Hiai, and Cabanal-Duvillar}

\begin{problem}
Given a tracial state $\tau$ corresponding
to a free stochastic process, does there exist a sequence of tracial
states $\tau_{n}\rightarrow\tau$ with $\chi_{p}^{*}(\tau_{n})\rightarrow\chi_{p}^{*}(\tau)$
where $\tau_{n}$ corresponds to the process $dA_{i}(t)=dS_{i}(t)+k_{t}(A_{1}(s),\ldots,A_{m}(s))_{s\leq t}dt$
with $k_{t}$ stepwise constant in $s$, and $\chi_{p}^{*}$ denotes
the quantity $\chi^{*}$ defined for processes in the paper of Guionnet
and Cabanal-Duvillard.
\end{problem}

\begin{problem}
In the one variable case, if $A(t)$ follows
a process $dA(t)=dS(t)+k_{t}(A(s))_{s\leq t}$ then replacing $A(t)$
with $A(t)+C_{\epsilon}$ (with $C$ having Cauchy distribution and
free from $A(t)$) then $k_{t}$ is replaced by $k_{t}^{\epsilon}=\tau\left(k_{t}\vert A(t)+C_{\epsilon}\right)$.
Thus, $k_{t}^{\epsilon}$ is smooth. Is there an analog of this smoothing
in the several-variable case?
\end{problem}

\begin{problem}
We know that if $f:\mathbb{R}\rightarrow\mathbb{R}$
and $A$ is an $n\times n$ Hermitian random matrix, then there exists
   a random matrix $C_{\epsilon}$ with Cauchy
distribution such that $\mathbb{E}f(A+C_{\epsilon})=P_{\epsilon}f(A)$
with $P_{\epsilon}f(x)=\int\frac{f(y)}{(y-x)^{2}+i\epsilon^{2}}dy$
the usual Cauchy (Poisson) kernel.    Can this be
done for several variables?
\end{problem}

\begin{problem}
 Given $x_{1},\ldots,x_{m}\in(\mathcal{A},\tau)$
a tracial unital vN algebra, do the conjugate variables belong to
the $L^{2}$ closure of cyclic gradient space? i.e. do there exist
$H_{k}\in\mathbb{C}\left\langle \alpha_{1},\ldots,\alpha_{m}\right\rangle $
such that $\mathcal{J}(x_{i})=\lim_{k}D_{i}H_{k}$ where $\partial_{x_{i}}:L^{2}(\mathcal{A},\tau)\rightarrow L^{2}(\mathcal{A},\tau)\otimes L^{2}(\mathcal{A},\tau)$
by $x_{j}\mapsto\delta_{ij}1\otimes1$ as a densely defined operator,
$\mathcal{J}(x_{i})=\partial_{x_{i}}^{*}(1\otimes1)$, and $D_{i}=m\circ\partial_{x_{i}}$
($m$ is the flip-multiplication $x\otimes y\mapsto yx$).
\end{problem}

\begin{problem}
 Does the change of variables
formula for $\chi$ also hold for $\chi^{*}$?
\end{problem}

\begin{problem}
 Is there a change of variables formula for
processes? i.e. suppose that we start with random variables $x_{1},\ldots,x_{m}\in(\mathcal{A},\tau)$
which can be reached by a process $dA_{i}(t)=dS_{i}(t)+k_{t}(A_{1}(s),\ldots,A_{m}(s))_{s\leq t}$,
$\mu_{A_{1}(1),\ldots,A_{m}(1)}=\mu_{x_{1},\ldots,x_{m}}$. We define
new random variables via functional calculus $y_{1}=f_{1}(x_{1},\ldots,x_{m}),\ldots,y_{m}=f_{m}(x_{1},\ldots,x_{m})$.
Can we apply a function $P$ to $k_{t}$ to get $dB_{i}(t)=dS_{i}(t)+P(k_{t}(B_{1}(s),\ldots,B_{m}(s))_{s\leq t})$
such that $\mu_{B_{1}(1),\ldots,B_{m}(1)}=\mu_{y_{1},\ldots,y_{m}}$.
\end{problem}


\begin{problem} Open problem:  Can we replace $\limsup$ with $\liminf$
in the microstates definition of the free entropy $\chi$?
\end{problem}

\begin{problem}
Hiai introduced the free pressure $\pi_R(h)$ for a self-adjoint element
(regarded as a free hamiltonian) $h$ of the universal free product
$C^*$-algebra $\mathcal{A}^{(n)}=\bigstar_{i=1}^nC([-R,R])$, and defined
a free entropy-like quantity $\eta_R(\tau)$ of a tracial state
$\tau\in TS(\mathcal{A}^{(n)})$. The inequality $\eta_R(\tau)\ge\chi(\tau)$
holds. $\tau$ is called an equillibrium tracial state with respect to $h$
if the variational equality $\eta_R(\tau)=\tau(h)+\pi_R(h)$ holds. Such a
$\tau$ always exists for each $h$. For which $h$ there is a unique equilibrium
tracial state? A way to prove this is the free transportation inequality.
\end{problem}

\begin{problemblock}

It was recently shown by Guionnet and Maurel-Segala
that for the vN algebra $(\mathcal{A},\tau)$ generated by $m$ free
semicirculars, \[
\sup_{\tau\in\mathcal{TS}(\mathcal{A})}\left\{ \chi(\tau)-\tau(\sum t_{i}q_{i})\right\} =\sum_{p_{1},\ldots,p_{m}}\prod_{k_{1},\ldots,k_{m}}\frac{(t_{i})^{p_{i}}}{k_{i}!}C(q,k_{1},\ldots,k_{m})\]
 where $C(q,k_{1},\ldots,k_{m})$ enumerated planar maps with colored
edges and vertices of types $q,k_{1},\ldots,k_{m}$.

\begin{problem}
Is there a similar
interpretation for the non-microstates analog \[
\sup_{\tau\in\mathcal{TS}(\mathcal{A})}\left\{ \chi^{*}(\tau)-\tau(\sum t_{i}q_{i})\right\} \textrm{?}\]\end{problem}

\end{problemblock}



\section{Free von Neumann Algebra}
\by{Dykema, Ricard}

\begin{problem}
Given $A,B$ free group factors with a
common diffuse subalgebra $D\subset A,B$, what conditions on $A,B,D$
guarantee that $A\mathop{\star}_{D}B$ is a free group factor?\end{problem}

\begin{problem}
For a regular weakly-rigid (in the sense of
Popa) subalgebra of a von Neumann algebra, is the free entropy dimension
$\leq1$?
\end{problem}

\begin{problemblock}
\begin{problem}
\textbf{\underbar{Open Problem}}: For generators $\gamma_{1},\ldots,\gamma_{n}\in\Gamma$
with the first $L^{2}$ -Betti number $\beta_{1}(\Gamma)$ large,
is the microstates free entropy dimension of this family of generators
large?
\end{problem}

\begin{comment}This is known for the non-microstates free entropy dimension
[work of Mineyev-Shlyakhtenko]
\end{comment}
\end{problemblock}

\begin{problem}
 Consider $\Delta=\sum_{i=1}^{m}\partial_{x_{i}}^{*}\partial_{x_{i}}$
and the corresponding completely positive map $\varphi_{t}=\exp(-t\Delta)$,
where $(x_{1},\ldots,x_{m})$ have finite Free Fisher Information.
Can $\varphi_{t}$ converge uniformly to the identity map on the unit
ball of $W^{*}(x_{1},\ldots,x_{m})$? If no, it follows that the von
Neumann algebra generated by $(x_{1},\ldots,x_{m})$ is not weakly
rigid if it is non-hyperfinite.
\end{problem}

% the next 3 problem blocks form a unit, but I was
% not sure how to indicate this.

\begin{problemblock}
 Let $\Gamma_{q,n}=W^*(s_q(g)=l(g)+l(g)^* |g\in \mathcal H_\mathbb R)$ with 
$n=\rm{ dim} \mathcal H_\mathbb R$, $-1<q<1$ 
be the von Neumann algebra generated by 
fields operators acting on a $q$-deformed Fock space
\begin{problem}
Does $\Gamma_{q,n}$
depend on $q$?
\end{problem}

A way to approach this question could come from
 the following observation. In the free case, $q=0$, 
the natural orthormal basis of 
the Fock space  consists of vectors 
$e_{\underline i}=e^{\otimes\alpha_1}_{i_1}\otimes ...
\otimes e^{\otimes\alpha_k}_{i_k}$ with $i_1\neq ...\neq i_k$ and 
$\alpha_1>0$. This basis can be recoved from the algebra as 
$e_{\underline i}=T_{\alpha_1}(s_0(e_1))...T_{\alpha_k}(s_0(e_q))\Omega$, 
where $T_k$ are Chebytchev polynomials. It 
would be interesting to find an analogue for these formulas in the general 
case and to unterstand the underlying combinatorics.

 \end{problemblock}

\begin{problemblock}

The $q$-deformation leads to the commutation relations $l(e)^*l(f)=ql(f)l(e)^*+\langle f,e \rangle Id$. Instead consider themore general relations
$l(e_i)^*l(e_j)=\sum_{s,t} t^{s,t}_{i,j} l(e_s)l(e_t)^*+\delta_{i,j} Id$.

\begin{problem}
When does the $C^*$-algebra generated by these operators is an extension of a 
Cuntz algebra by compacts ? When does the fields operators associated to them 
produce a type $II_1$ factor ?
\end{problem}
\end{problemblock}


\begin{problemblock}

Consider the projection $P_k$ from $\Gamma_{q,n}$ to its subspace
consisting of $x$ such that $x.\Omega$ has length at most $k$ in the Fock 
space.
\begin{problem}
 Is $\|P_k\|_{cb}$ polynomially bounded in $k$ ?
\end{problem}
This would prove the 
CBAP for the associated $L_p$ spaces $(1<p<\infty)$ and the exactness of the 
$C^*$-algebra generated by $q$-gaussians.
 \end{problemblock}\begin{problem}
To prove the existence of an embedding $\Gamma_{q,n}\to \mathcal R^\omega$, 
one 
uses Speicher's central limit theorem. In this procedure, 
is it possible to find explictely uniformly bounded matrix whose mixed moments
approach those of $q$-gaussians ? More precisely, let $c_{i,j}$ be unitary
generators of the CAR-algebra (or $-1$-gaussians), are the matrices $\frac 1
{\sqrt n}[c_{i,j}]_{i,j\le n}$ uniformly bounded ? \end{problem}
\begin{problemblock}
\begin{problem}
 For the random matrix model $\exp(-nTr(p(A_{1},A_{1}^{*},\ldots,A_{m},A_{m}^{*}))$
we know that the conjugate variables satisfy $\mathcal{J}_{i}=\mathcal{D}_{i}P$.
Is the operator $\exp(-t\sum\partial_{j}^{*}\partial_{j})$ compact
in the limit $n\rightarrow\infty$ (where $\partial_{j}$ is Voiculescu's
partial difference quotient on the limit algebra with respect to the
limit of $A_{j}$)?
\end{problem}\begin{comment}As a starting point, consider $P=\sum A_{i}^{2}+\sum t_{i}q_{i}(A_{1},\ldots,A_{m})$
where Guionnet and Maurel-Segala have shown convergence of the model.
\end{comment}
 \end{problemblock}

\section{Focus Group on Free Entropy}
The problems in this section arose during a discussion group
on Free Entropy during the 3rd day of the AIM workshop.

\begin{problemblock}
Let
 \[
\delta^{*}=n-\limsup_{t\downarrow0}\frac{\chi^{*}(x_{1}+\sqrt{t}s_{1},\ldots,x_{n}+\sqrt{t}s_{m})}{\log t^{1/2}}\]
 and\[
\delta^{\star}=n-\limsup_{t\rightarrow0}\sum_{i=1}^{n}t\Phi^{*}(x_{1}+\sqrt{t}s_{1},\ldots x_{m}+\sqrt{t}s_{m})\textrm{.}\]

%The following construction put everything later into italics?
%\problem{Does $\delta^{*}=\delta^{\star}$?}
\begin{problem}{Does $\delta^{*}=\delta^{\star}$?}
\end{problem}

\end{problemblock}

% the surrounding {} seem to be necessary here, or the the later fonts go wrong
\begin{problem} What is the non-microstates analogue
of free entropy in the presence,
$\chi(x_{1},\ldots,x_{n}:y_{1},\ldots,y_{n})$?          
\end{problem}

\section{Focus Group on Operator Theory}
The problems in this section arose during a discussion group
on Operator Theory during the 3rd day of the AIM workshop.

\begin{problem}What is the boundary behavior of the subordination
functions which appear in free convolution of operator-valued random
variables?
\end{problem}

\begin{problemblock}
\begin{problem}What are examples/conditions for freely strongly
unimodal variables, i.e. unimodal random variables that when freely
convolved with a unimodal variables remain unimodal?
\end{problem}
\begin{comment}Unimodal means
that the law of the random variable has a smooth density with a unique
maximum; example: Gaussian law or the semicircle law.
\end{comment}
\end{problemblock}

{
\begin{problem} More specifically, if $\mu,\nu$ are symmetric
unimodal distribution, is $\mu\boxplus\nu$ unimodal?
\end{problem}
}

\section{Invariant Subspaces for an Operator}
\by{Haagerup}

\begin{problem}
 Let $x,y$ be two free circular elements, and
let $S,T$ be two operators in a II$_{1}$ factor, which is free from
$x,y$. In the Haagerup-Schultz estimate      
   $$(\star\star)\qquad\left\Vert (S+xy^{-1})^{-1}-(T+xy^{-1})^{-1}\right\Vert _{p}\leq c(p)\left\Vert S-T\right\Vert _{p}<\infty$$
with $0<p<\frac{2}{3}$,        
       can one use $x$ instead of $xy^{-1}$?\end{problem}

\begin{problemblock}
(Brown measure of unbounded operators): As
defined by (Haagerup and Schultz), $\Delta(T)$   makes sense
for $T\in M^{\Delta}$ where $M^{\Delta}=\left\{ T\in\tilde{M}\vert\int_{0}^{\infty}\log t\, d\mu_{T}(t)<\infty\right\} $.
Then $\Delta(T)=\exp(\int_{0}^{\infty}\log t\, d\mu_{T}(t))\in[0,\infty]$.
\begin{problem}
Can one make sense of $\mu_{T}$ for such unbounded $T$?
\end{problem}
\end{problemblock}

\section{Free Group Factor}\by{Ozawa}

% conjecture within a problem.  
\begin{problemblock}
\begin{conjecture}If $\mathcal{H}$ an $M\textrm{-}M$ bimodule
$M=L\mathbb{F}_{n}$, and $_{M}\mathcal{H}_{M}\preceq L^{2}M\otimes L^{2}M$,
(weak containment) then \[
\textrm{Hom}(_{M}\mathcal{H}\mathop{\otimes}_{M}\mathcal{H}\mathop{\otimes}_{M}\mathcal{H}_{M},L^{2}M\otimes L^{2}M)\neq0\textrm{.}\]
\end{conjecture}


Note that the assumption of weak containment is equivalent that the
map\[
x\otimes y\mapsto(\lambda(x)\rho(y):\mathcal{H}_{M}\ni h\mapsto xhy)\in B(_{M}\mathcal{H}_{M})\]
is continuous for the min-tensor product on $M\otimes M$. Examples
of bimodules with this property come from the basic construction\[
_{M}\mathcal{H}_{M}=M\otimes_{A}M\]
over a hyperfinite subalgebra $A\subset M$.

\end{problemblock}

\section{Combinatorics of Random Matrix Models}

The material in this section arose during a discussion group on
the 4th day of the AIM workshop.

% note use of \by in theorem environment

Given random matrices $A_{n}$ and $B_{n}$ with corresponding measures
$\mu_{A_{n}}$ and $\mu_{B_{n}}$ on $M_{n}(\mathbb{C})$, we define
their Itzykson-Zuber integral as\[
IZ(A_{n},B_{n})=\int\exp(-nTr(AU^{*}BU))d\mu_{A_{n}}(A)d\mu_{B_{n}}(B)\textrm{.}\]
 \begin{theorem} \by{Guionnet and Zeitouni} If $\left\Vert A_{n}\right\Vert <c$,
$\left\Vert B_{n}\right\Vert <c$ then $IZ(A_{n},B_{n})\sim\exp(-n\psi)$.
\end{theorem}

\begin{problem}
 There is another result that states that \[
\frac{\partial^{n}}{\partial t^{n}}\log IZ(tA_{n},B_{n})\vert_{t=0}\textrm{ converges.}\]
 Does this expression match $\psi$ above? Can we extend Guionnet
and Zeitouni's result to complex parameters?\end{problem}

\begin{problem}
Extend the model $\exp(-nTr(P(A_{1},\ldots,A_{m})+\frac{1}{2}\sum_{i=1}^{m}A_{i}^{2}))dA_{1}\ldots dA_{m}$
of Guionnet and Maurel-Segala to non-selfadjoint $P$ (i.e.polynomials
with complex coefficients).
\end{problem}

\begin{problem}
 Is there a combinatorial interpretation of
free cumulants in terms of enumeration of maps and operations on maps?
\end{problem}

\begin{problemblock}
Consider the spherical integrals
$$I_n(z,E_n):=\int\exp\{n{\rm tr}(UD_nU^*E_n)\}d_{m_n}(U),$$
where $D_n={\rm diag(z,0,0,\dots,0)}$, $z\in\mathbb C$, and
$E_n$ is a sequence of $n\times n$ selfadjoint (diagonal) matrices,
with spectrum uniformly bounded in $n$, and  converging in distribution to
$\mu_E$

The sequence of functions of $z$
$$f_n(z)=\partial_z\frac1n\log I_n(z,E_n),$$
has been shown by Guionnet and Maida to converge to $R_{\mu_E}(z)$ for
$|z|$ small enough.
\begin{problem}
 What is the largest domain in the complex plane on which
this convergence takes place? If $\mu_E$ is $\boxplus$-infinitely
divisible, is the convergence happening on all the upper half-plane? Is there any possible generalization to measures with noncompact support?
\end{problem}
\begin{comment}One could probably approach this problem by trying to study the normality
of the family/sequence $f_n$.
\end{comment}
\end{problemblock}

\section{Invariant Subspaces}
The questions in this section arose during the focus group on
Invariant Subspaces during the 4th day of the workshop.


If $M$ is a $\textrm{II}_{1}$ factor, $T_{1},\ldots,T_{n}\in M$,
$[T_{i},T_{j}]=0$, then we have the {}``Brown Measure'' defined
as the unique measure on $\mathbb{C}^{n}$ such that \[
(\star)\qquad\log\Delta(1-\sum\alpha_{i}T_{i})=\int\log(1-\sum\alpha_{i}\zeta_{i})d\mu_{T_{!},\ldots,T_{n}}(\zeta_{1},\ldots,\zeta_{n})\textrm{.}\]

\begin{problem}
 Is $\textrm{supp}\mu_{T_{1},\ldots,T_{n}}\subset\sigma(T_{1},\ldots,T_{n})$,
the Taylor spectrum of $T_{1},\ldots,T_{n}$?
\end{problem}

\begin{problem}
 Which functions on $\mathbb{C}^{n}$ have an
integral representation as in $(\star)$o
\end{problem}

\begin{problemblock}
$M$ a $\textrm{II}_{1}$ factor and $T\in M$.
Define\[
K(T,r)=\left\{ \xi\in\mathcal{H}\vert\exists\xi_{n}\in\mathcal{H}\textrm{ s.t. }\left\Vert \xi_{n}-\xi\right\Vert _{2}\rightarrow0\textrm{ and }\limsup\left\Vert T^{n}\xi_{n}\right\Vert ^{1/n}\rightarrow0\right\} \textrm{,}\]
 \[
\textrm{and }E(T,r)=\left\{ \xi\in\mathcal{H}\vert\limsup\left\Vert T^{n}\xi_{n}\right\Vert ^{1/n}\rightarrow0\right\} \textrm{.}\]
\begin{problem}
Does $K(T,r)=E(T,r)$?
\end{problem}
\begin{comment}The $DT$ quasinilpotent operator may be a
counterexample.
\end{comment}
\end{problemblock}

\begin{problem}
 Let $c$ be a circular element ($\sigma(c)=\bar{\mathbb{D}}$),
and let $f\in C^{\infty}(\mathbb{C})$. Can we make sense of $f(c)$
as an (unbounded) operator affiliated with $\left\{ c\right\} ^{\prime\prime}$?
\end{problem}

\begin{problem}
 Let $(\Gamma,\tau)$ be a $\textrm{II}_{1}$
factor, $T\in\Gamma$, $\mu_{T}=\delta_{0}$. Does $T$ have a non-trivial
invariant subspace affiliated with $\Gamma$?
\end{problem}

% the following should be converted to a problemblock
\begin{problem}
 Let $B_{c}$be a band limited operator obtained
from $c$ a circular element, and let $D$ be the band limited operator
obtained from the identity. Then $D$ is uniformly distributed on
$[0,1]$ and $\star$-free from $\left\{ B_{c},B_{c}^{*}\right\} $.

Is $D\in W^{*}(B_{c})$? Or is $W^{*}(B_{c})=L\mathbb{F}_{t}$ with
$t=1+2c(1-\frac{c}{2})$?
\end{problem}

\section{Infinite Divisibility}\by {Nica}

\begin{problemblock}
 Given $x_{1},\ldots,x_{k}$ and $y_{1},\ldots,y_{k}$
in a vNa such that $\left\{ x_{1},\ldots,x_{k}\right\} $ is tensor-independent
of $\left\{ y_{1},\ldots,y_{k}\right\} $ and such that $\mu_{x_{1},\ldots,x_{k}},\nu_{y_{1},\ldots,y_{k}}$
are freely infinitely divisible, we can apply the Fourier transform
to get the power-series of the classical convolution of $\mu_{x_{1},\ldots,x_{k}}$
and $\nu_{y_{1},\ldots,y_{k}}$.
\begin{problem}
How do such power-series relate to
the noncommutative power series obtained from free convolution?
\end{problem}
In
other words how does the set of classically obtainable power-series
relate to the set of freely obtainable power-series?
\end{problemblock}

\begin{problemblock}
\begin{problem}
 Can we make sense of the R-transform for $x_{1},x_{2}$
unbounded (power-series are insufficient to encode all the information)?\end{problem}
Easier question is for infinitely divisible unbounded operators.
\end{problemblock}

\begin{problem} If $c$ is unbounded R-diagonal, what is the
R-transform of $c,c^{*}$?
\end{problem}

\section{ Dirichlet Forms, from Classical to Quantum}
This section comes from a focus group on the 5th day of the AIM workshop.

\begin{problemblock}
\begin{problem}
 For the $q$-deformed semicircular, the analogue
of $\partial^{*}\partial$ exists (it is the number operator). 
     Describe explicitly the associated $\partial$.
\end{problem}
It exists by the work of Sauvageot.
\end{problemblock}

\begin{problem}
 More generally, given a negative definite
function on a group $\Gamma$ (i.e. a Dirichlet form), we know it
gives a representation by affine actions on $L^{2}\Gamma$. When
is it a multiple of the left regular representation? What conditions
on the negative definite function guarantee this?\end{problem}
\begin{problem}
 What conditions on a Dirichlet form $\delta^{*}\delta$
guarantee that the bimodule associated to $\delta$ embeds into $\bigoplus L^{2}N\otimes L^{2}N$? \end{problem}
\begin{problem}
What is the analogue of the Bakry-Emery criterion
in the noncommutative case? i.e. what is $\Gamma_{2}$ for noncommutative
Dirichlet forms?\end{problem}

\begin{problem}

 Let $\partial:M\to L^{2}(M)\bar{\otimes}L^{2}(M^{o})$
be a closable derivation, and let $\Delta=\partial^{*}\partial$,
$S_{t}=\exp(-t\Delta)$. If the semigroup $S_{t}$   
  converges uniformly to the identity in $\left\Vert \cdot\right\Vert _{2}$
on the unit ball, is the derivation inner when considered with values
in the algebra of unbounded operators affiliated to $M\bar{\otimes}M^{o}$?
\end{problem}

\end{document}

\begin{problemblock}

ZZZZ Lead-in to the second problem

\begin{problem}
ZZZZ Statement of second problem
\end{problem}

ZZZZ You can put words after the problem statement and before the first
ZZZZ comment, but you don't have to and most of the time it is better
ZZZZ to put the words in a comment.

\begin{comment}
ZZZZ First comment on the second problem.
\end{comment}

\begin{comment}
ZZZZ Second comment on the second problem.
\end{comment}

\end{problemblock}

ZZZZ just keep repeating the problemblock structure


\end{document}
